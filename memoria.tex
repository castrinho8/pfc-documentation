\documentclass[a4paper,spanish,10p,titlepage]{report}

% Se cargan los paquetes necesarios
\usepackage[utf8]{inputenc}
\usepackage{graphicx}
\usepackage{helvet}
\usepackage[T1]{fontenc}
\usepackage{verbatim}
\usepackage{indentfirst}
\usepackage{color}
\usepackage[x11names]{xcolor}
\usepackage{hyperref}
\usepackage{multirow}
\usepackage{wasysym}
\usepackage{pstricks}
\usepackage{courier}
\usepackage{caption}
\usepackage{listings}
\usepackage{pifont}
\usepackage{eurosym}
\usepackage{color}
\usepackage{wrapfig}
\usepackage{appendix}
\usepackage[left=3.2cm,top=3cm,right=3.5cm,bottom=3cm]{geometry} 

\DeclareCaptionFont{white}{\color{white}}
\DeclareCaptionFormat{listing}{\colorbox{gray}{\parbox{\textwidth}{#1#2#3}}}
\captionsetup[lstlisting]{format=listing,labelfont=white,textfont=white}

\hypersetup{
  colorlinks=true,
  linkcolor=blue
}
\definecolor{green}{RGB}{0,153,0}
\definecolor{red}{RGB}{153,0,0}

\newcommand{\HRule}{\rule{\linewidth}{0.5mm}}

\renewcommand{\appendixname}{Anexos}
\renewcommand{\appendixtocname}{Anexos}
\renewcommand{\appendixpagename}{Anexos}


\begin{document}

\title{\Huge{Integración de sistema Web y Android para la gestión de actas 
electrónicas.} \\ \small{}}
\author{Pablo Castro Valiño \\ \small{(pablo.castro1@udc.es)}
}

\date{15 de Febrero de 2015}

\renewcommand{\tablename}{Tabla}

\maketitle

\tableofcontents
\newpage

%Indice de figuras
%Indice de cuadros

\chapter{Introducción}


\section{Origen y ecosistema del proyecto}
    
    \subsection{Motivación}
    Actualmente existe una enorme falta de informatización en la gestión de 
competiciones deportivas por parte de federaciones y asociaciones que organizan dichas 
competiciones.

Los trámites se realizan manualmente y se tardan días publicar los resultados de una 
nueva jornada siguiendo el siguiente proceso:
    \begin{enumerate}
     \item La federación crea un calendario de encuentros que publica en su web.
     \item Un árbitro recoge un \textbf{acta} y la translada al campo o a la pista en la 
que se disputa el partido que debe arbitrar.
     \item Cada jugador lleva su \textbf{ficha identificativa} al partido.
     \item El árbitro cubre el \textbf{acta} con los datos de todas las \textbf{fichas} 
de los jugadores.
     \item Durante el encuentro, el árbitro de mesa o el 4º árbitro rellena el 
\textbf{acta} manualmente, cubriendo las estadísticas del encuentro.
     \item El \textbf{acta} es firmada por el árbitro y un representante de cada equipo.
     \item Cada club guarda una copia del \textbf{acta} y el árbitro transalada la suya a 
la federación.
     \item La federación revisa el \textbf{acta}, comprobando que los jugadores que han 
jugado no estaban sancionados, si pertenecen al equipo y copiando todos los datos 
recogidos, a la aplicación de gestión de la que dispongan.
    \end{enumerate}

    Es por ello que hemos decidido crear una aplicación para mejorar la gestión de las 
competiciones de las federaciones deportivas a través de la informatización, prácticamente 
completa, de sus tareas administrativas, desde la generación de actas en los partidos 
hasta la inscripción online en dichas competiciones por parte de los clubes.
  
    El proyecto a desarrollar como \emph{Proyecto de Fin de Carrera} será el eje 
central de dicho sistema, un elemento básico y diferenciador que se integrará en la 
aplicación de gestión, facilitando la gestión electrónica de actas y fichas deportivas.

  \subsection{Necesidades que pretende cubrir}
  
    \paragraph{Acceso a datos deportivos actualizados a diario.}
    Es poco habitual ver los datos de los partidos actualizados a diario, incluso 
semanalmente, lo 
cual dificulta seguir la competición y provoca la pérdida de interés por parte de 
jugadores y 
equipos.
  
    Esta plataforma servirá como lugar central de acceso a toda esta información que 
habitualmente se encuentra poco actualiza y totalmente dispersada por las diversas 
páginas web de las federaciones.
    
    \paragraph{Gestión de actas.}
    Las actas contienen una información muy valiosa que no se está explotando y su gestión 
se 
encuentra muy lejos de estar tecnológicamente actualizada.
Cambiando el papel por la electrónica e informatizando el proceso se podrá reunir esta 
valiosa información, publicar datos con mayor presteza y proporcionarla de una forma que 
pueda ser tremendamente util para entrenadores y ojeadores.
    
    \paragraph{Trámites manuales.}
    Entre otros trámites que se encuentran muy atrasados tecnológicamente están las 
inscripciones en las competiciones o los aplazamientos de partidos en los que el equipo 
interesado 
debe contactar con la federación para obtener el contacto del equipo rival. 

Una vez ambos clubes llegan a un acuerdo en el aplazamiento del partido, deben enviar por 
correo o 
fax un modelo conforme están de acuerdo en el aplazamiento y una vez la 
federación reciba ambos modelos debe avisar al árbitro correspondiente.

Todo el proceso se realiza de manera manual, tardando entre 2 y 3 días, algo que podría 
realizarse 
de forma prácticamente instantánea a través del sistema propuesto.  

\clearpage

% -------------------------------------

\chapter{Fundamentos tecnolóxicos}
  \section{Linguaxes e frameworks empregados}
  \section{Repositorios de código}
  \section{Ferramentas de xestión}
  \section{Ferramentas documentáis}

\clearpage

% -------------------------------------

\chapter{Metodoloxía}

  \section{eXtreme Programming}

  \section{Adaptación da metodoloxía}
    \subsection{Reunións semanáis}
    
    \subsection{Reunións diaria}
    
    \subsection{Sprints}
    Cada dúas semanas.

    \subsection{Releases}
    4 Entregas pequenas.

    \subsection{Fluxo de traballo}
    PR en github con code review e integración continua

\clearpage

% -------------------------------------

\chapter{Planificación e seguimento}

  \section{MVP}
    \paragraph{Planificación temporal}
    \paragraph{Definición da iteración}
    \paragraph{Feedback}
    \paragraph{Tarefas e seguimento}
  \section{1ª iteración. APP mobil unhosted}
  \section{2ª iteración. API Rest}
  \section{3ª iteración. Xestión de usuarios}
  \section{4ª iteración. Actas offline}

\clearpage

% -------------------------------------

\chapter{Análise de requisitos globáis}

  Neste capítulo exporemos o proceso de análise de requisitos para o desenvolvemento do 
proxecto.

  \section{Consultas a xestores de federacións}
  Para a realización deste apartado decidíuse consultar con diversas asociacións 
deportivas e federacións das que obter feedback acerca das necesidades que 
actualmente están a demandar co fin de obter certas funcionalidades a implementar e 
obtendo incluso a priorización segundo as necesidades máis urxentes.


  \section{Peticións obtidas}
  
  \paragraph{Cubrir acta en tempo real}
  \paragraph{Permisos}
  \paragraph{Sincronización} Coa web da federación
  \paragraph{Persoas convocadas}
  \paragraph{Eventos} Mostralos e engadilos
  \paragraph{Motivación dun evento} Por exemplo, unha tarxeta amarella. Incluso motivos 
por defecto
  \paragraph{Editar dorsal dun xogador}
  \paragraph{Persoa con varios roles} Xogador + entrenador
  
  \clearpage

  \section{Requisitos fináis}

  
  \subsection{Usuarios}
    \paragraph{Facer login e logout} OAuth
    \paragraph{Permisos para edición de actas}
    \paragraph{Mostrar perfil do usuario}

  \subsection{Listar actas}
    \paragraph{Visualizar próximas actas a cubrir dun árbitro}
    \paragraph{Visualizar actas cubertas dun árbitro}
    \paragraph{Actualización automática de actas descargadas ante modificacións}

  \subsection{Visualizar actas}
    \paragraph{Listar o personal dun equipo}
    \paragraph{Listar os xogadores dun equipo}
    \paragraph{Visualizar datos xeneráis dun acta} (resultados, faltas, etc)
    \paragraph{Visualizar eventos dun acta} (dúas vistas)

  \subsection{Xeración de actas}
    \paragraph{Xeración automática}
    \paragraph{Xeración de actas offline}

  \subsection{Modificación de actas}
    \paragraph{Modificación de propiedades da acta} (Federación)
    \paragraph{Convocar un xogador}
    \paragraph{Engadir un xogador que non está no equipo}
    \paragraph{Editar o dorsal dun xogador}
    \paragraph{Engadir eventos deportivos}
    \paragraph{Poder engadir motivos dun evento xerado}
    \paragraph{Cambiar de parte}
    \paragraph{Modificar o tempo}
    \paragraph{Engadir observacións na acta}
    \paragraph{Asinar a acta} (árbitro e xogadores)


\end{document}




