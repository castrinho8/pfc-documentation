\chapter{Fundamentos tecnolóxicos}
\minitoc
% \label{chap:Fundamentostecnoloxicos}
% \vspace{0.5cm}

%%%%%%%%%%%%%%%%%%%%%%%%%%%%%%%%%%%%%%%%%%%%%%%%%%%%%%%%%%%%%%%%%%%%%%%%%%%%%%%%
% Objetivo:                        %
%%%%%%%%%%%%%%%%%%%%%%%%%%%%%%%%%%%%%%%%%%%%%%%%%%%%%%%%%%%%%%%%%%%%%%%%%%%%%%%%

  \lettrine{N}{este} capítulo mostraranse as diversas tecnoloxías que foron empregadas
durante o desenvolvemento do proxecto así como ferramentas de xestión e documentais,
todas, tecnoloxías Software Libre.

\santiagosays{Confundiume que vai primeiro a metodoloxía pero sin embargo están
  numerados os ficheiros fonte \texttt{.tex} ao revés}

  \section{Linguaxes e frameworks empregados}

  \todo[inline]{Añadir página web de cada tecnología: EJ: ReactJS ([1]) y el link en la
bibliografía}
\santiagosays{Idealmente, a referencia sería a un libro. Se hai algún
  libro ou libros que sexan boas referencias, teñen máis relevancia as
  publicacións formais ca as páxinas web. Salvo que queiras ser un
  radical que vai contra corrente no asunto, e defender a capa e
  espada que as autopublicacións no século 21 son publicacións. Se
  queres facer iso podes incluso poñer unha sección ao respecto
  explicando que é unha decisión editorial :-)}

\santiagosays{En xeral, partes desta sección deberían de trasladarse
  ao state of the art, onde se explicarían \emph{que} son, e aquí
  habería que mencionar \emph{porqué} se empregan as que se decidiron
  utilizar, porqué teñen sentido e coherencia xuntas, e todo iso. En
  breve: un capítulo para decir qué hai no mercado; e outro para
  decir, do que hai, que se escolleu como fundamento, e porqué. Se
  queres mais interacción sobre isto podemos falalo, claro}

  \santiagosays{Introduce a sección cun miniparágrafo}

  \begin{description}
   \item [ReactJS] React\santiagosays[]{Cita requerida} é unha librería de Javascript\santiagosays{Non falas de Javascript ata agora; pode ser que se entenda que é ``coñecemento común'', pero aínda así eu engadiría cita ao estándar (ECMA)} para a creación de Single Page
Aplications (SPAs), permitindo crear aplicacións no frontend\santiagosays[]{Unha aplicación non ten sempre frontend e backend?} de forma sinxela a través de
diversos compoñentes que se agrupan e que permiten crear aplicacións multiplataforma.

   \item [Reflux] É unha implementación da arquitectura Flux impulsada por Facebook\santiagosays[]{Impulsada a arquitectura ou Reflux?} e que
permite un fluxo de datos unidireccional, en lugar do bidireccional que é habitual nas
aplicacións web.
    Permite tamén a creación de Stores nas que se mantén o estado das aplicacións e que
permite compartir dito estado entra os diversos compoñentes da aplicación.\santiagosays{Non contas moita cousa aquí, para ser un paradigma tan pouco usual hoxe en día, eu creo}

   \item [Jest] É unha ferramente sinxela creada sobre o framework de testing
para Jasvascript, Jasmine, que facilita a utilización de mocks e a creación de
tests unitarios.

   \item [Jed (i18n)] Jed é unha ferramenta para facilitar a internacionalización de
aplicacións Javascript utilizando o estándar Gettext e utilizando unha API moi sinxela.

   \item [React Router] Unha librería para o enrutado de aplicacións baseadas en ReactJS
proveendo unha API sinxela con funcionalidades de gran potencia como a carga preguiceira
de código ou o enrutado dinámico.

   \item [Material UI] Un conxunto de compoñentes para React que implementan o Material
Design impulsado por Google, unha nova linguaxe visual baseada na representación en 3D
dos obxectos que non deben intersecarse se non que a través de sombras para simular
diferentes profundidades, os obxectos debe superpoñerse uns sobre os outros.

  \end{description}

  \section{Bases de datos}

  \santiagosays{Introduce a sección cun miniparágrafo}

  \begin{description}
     \item [PouchDB] Unha base de datos NoSQL baseada en Javascript e inspirada
en CouchDB, pensada para facilitar o funcionamento de aplicacións web de forma
offline.

PouchDB permite almacenar os datos localmente no navegador web cando non hai
conexión a internet e sincronizar de forma sinxela ditos datos en remoto con
CouchDB e outros servidores compatibles.%
\santiagosays{PouchDB é unha base de datos non relacional \emph{implementada} en
  Javascript, pero porque non hai outro xeito. Implementa o esquema de
  replicación de CouchDB, o que a fai moi conveniente para ter unha copia local
  dos datos remotos, de xeito que permita unha experiencia máis fluida e
  \emph{seamlessly offline}. Como CouchDB emprega un paradigma distribuido,
  sincronizar datos dende ou cara unha implementación PouchDB resulta sinxelo.}

\item [CouchDB] \santiagosays[]{Cita requerida} É unha base de datos
  pensada para web que permite almacenar os datos en formato JSON e
  acceder aos mesmos a través dun navegador via HTTP, funcionando como
  unha API REST.\santiagosays{Que é REST? Eso é parte das tecnoloxías.
    Ademais, non é simplemente unha base de datos pensada para web: é
    unha base de datos non relacional, orientada a documentos, que
    implementa o paradigma Map-Reduce para a busca de elementos, dun
    xeito eficiente para a distribución en cluster/cloud/como lle
    queiras chamar.}

  Permite gran cantidade\santiagosays[]{Xuízo de valor} de funcionalidades como
  servir aplicacións directamente desde CouchDB así como un sistema de
  replicación incremental e de detección de conflictos. \ProTip{En lugar de
    calificar de ``gran cantidade'' e nomear algunha; cabe describilo como que
    \emph{ten funcionalidades \textbf{que habitualmente non forman parte dos
        sistemas de xestión de bases de datos}, como a replicación incremental,
      a detección de conflitos, ou o acceso mediante unha interface REST aos
      datos}.}

  \end{description}

  \section{Estándares de comunicación}

  \santiagosays{Introduce a sección cun miniparágrafo}

  \begin{description}
   \item [JSON] É un formato estándar\santiagosays[]{Cita requerida} para o intercambio de datos e que pola súa
simplicidade estase a impoñer como formato habitual por exemplo, para a comunicación con
APIs Rest e debido a súa similitude coa definición de obxectos en Javascript, permite que
sexa tremendamente sinxelo traballar con él dende esta linguaxe.
  \end{description}

  \section{Repositorios de código}

  \santiagosays{Introduce a sección cun miniparágrafo}

  \begin{description}
   \item [GitHub] Github é un repositorio de código que se está a convertir no lugar máis
importante de publicación de aplicacións Software Libre\santiagosays{En realidade, de código aberto; hai moito software libre en SourceForge aínda, e tamén moito en gitlab; e ten sentido porque estes últimos, ao igual que gitorious, son software libre tamén o propio servidor do servizo}
como o presente, sen ningún custe.
  Cómpre destacar que esta é a única ferramenta utilizada para o desenvolvemento do
proxecto que non é software libre pero si proporciona unha visibilidade de cara a
comunidade de gran importancia neste tipo de proxectos.
  \end{description}

  \section{Ferramentas de xestión}

  \santiagosays{Introduce a sección cun miniparágrafo}

  \begin{description}
   \item [OpenShift] É a plataforma na nube da empresa Red Hat que permite
realizar despregamentos de aplicacións de forma sinxela e basada nunha solución software
libre.\santiagosays[]{E porqué mola? (Engade esto ao final de cada item}
   \item [Git] \santiagosays[]{Cita requerida} É un sistema de control de versións software libre de gran potencia\santiagosays[]{Que significa potencia?} e
utilizado en millóns de proxectos que aporta unha versatilidade enorme ao ser
distribuida, permitindo traballar incluso de forma offline.
   \item [Gulp] Un sistema que permite a automatización de tarefas durante o
desenvolvemento de aplicacións como por exemplo compilar automáticamente o código
Javascript escrito na súa última versión á versión máis antiga para que poida ser
executada por calquera navegador web.
   \item [Redmine] É unha ferramenta de xestión de proxectos flexible, multiplataforma e
software libre con diversos plugins para facilitar a planificación de iteracións e
traballar con metodoloxías áxiles de desenvolvemento.
   \item [Travis CI] É unha ferramenta de integración continua que permite automatizar a
execución de tests e o despregamento automático de código. Ademais intégrase con GitHub de xeito que pode executarse automaticamente ao se subiren cambios novos, polo que resulta moi sinxelo automatizar estas tarefas.
   \item [Atom] Un editor de texto software libre deseñado inicialmente por GitHub de
gran\santiagosays[]{Xuízo de valor} potencia\santiagosays[]{De novo, que é potencia?} e extensibilidade gracias a un sinxelo sistema de plugins. Ademáis é un
editor impulsado por Facebook (creadores de ReactJS) para facilitar o traballo con esta
tecnoloxía.\santiagosays{Ademais, no ecosistema ReactJS, existen componentes para facilitar a edición, elaboradas polos mesmos autores ca o propio React}

  \end{description}

  \section{Ferramentas documentáis}

  \santiagosays{Introduce a sección cun miniparágrafo}

  \begin{description}
  \item [LaTeX] Un sistema para a composición de documentos que inclúe
    diversas funcionalidades para a edición de textos científicos
    ou técnicos, moi adecuado para este proxecto e que xenera
    documentos de gran calidade.\santiagosays[]{De novo, gran calidade
      é un xuízo de valor.}%
    \santiagosays{Para que se usou? Para a elaboración desta memoria? Escríbeo ;-) \\
      Nun futuro se queres ter manual de usuario da aplicación, en que
      se vai a facer, en LaTeX? Texinfo? Comentarios no código? :-)}
   \item [Dia] É unha aplicación para a creación de diagramas entre os que se atopan os
diagramas UML e que permite a exportación dos mesmos a imaxes vectoriais.
  \end{description}



%%% Local Variables:
%%% mode: latex
%%% TeX-master: "../root"
%%% End:
