\chapter{Fundamentos tecnolóxicos}
\minitoc
% \label{chap:Fundamentostecnoloxicos}
% \vspace{0.5cm}

%%%%%%%%%%%%%%%%%%%%%%%%%%%%%%%%%%%%%%%%%%%%%%%%%%%%%%%%%%%%%%%%%%%%%%%%%%%%%%%%
% Objetivo:                        %
%%%%%%%%%%%%%%%%%%%%%%%%%%%%%%%%%%%%%%%%%%%%%%%%%%%%%%%%%%%%%%%%%%%%%%%%%%%%%%%%

  \lettrine{N}{este} capítulo mostraranse as diversas tecnoloxías que foron empregadas 
durante o desenvolvemento do proxecto así como ferramentas de xestión e documentáis, 
todas, tecnoloxías Software Libre.

  \section{Linguaxes e frameworks empregados}

  \todo{Añadir página web de cada tecnología: EJ: ReactJS ([1]) y el link en la 
bibliografía}

  \begin{description}
   \item [ReactJS] React é unha librería de Javascript para a creación de Single Page 
Aplications (SPAs), permitindo crear aplicacións no frontend de forma sinxela a través de 
diversos compoñentes que se agrupan e que permiten crear aplicacións multiplataforma.

   \item [Reflux] É unha implementación da arquitectura Flux impulsada por Facebook e que 
permite un fío de datos unidireccional, en lugar do bidireccional que é habitual nas 
aplicacións web.
    Permite tamén a creación de Stores nas que se mantén o estado das aplicacións e que 
permite compartir dito estado entra os diversos compoñentes da aplicación.

   \item [Jest] É unha ferramente sinxela creada sobre o framework de testing 
para Jasvascript, Jasmine, que facilita a utilización de mocks e a creación de 
tests unitarios.

   \item [React Intl (i18n)] React Intl é unha ferramenta para facilitar a 
internacionalización de aplicacións Javascript e concretamente as 
aplicacións baseadas en React.

   \item [React Router] Unha librería para o enrutado de aplicacións baseadas en ReactJS 
proveendo unha API sinxela con funcionalidades de gran potencia como a carga preguiceira 
de código ou o enrutado dinámico.

   \item [Material UI] Un conxunto de compoñentes para React que implementan o Material 
Design impulsado por Google, unha nova linguaxe visual baseada na representación en 3D 
dos obxectos que non deben intersecarse se non que a través de sombras para simular 
diferentes profundidades, os obxectos debe superpoñerse uns sobre os outros.
   
  \end{description}

  \section{Bases de datos}

  \begin{description}
     \item [PouchDB] Unha base de datos NoSQL baseada en Javascript e inspirada 
en CouchDB, pensada para facilitar o funcionamento de aplicacións web de forma 
offline.

PouchDB permite almacenar os datos localmente no navegador web cando non 
hay conexión a internet e sincronizar de forma sinxela ditos datos en remoto con 
CouchDB e outros servidores compatibles.

   \item [CouchDB] É unha base de datos pensada para web que permite almacenar 
os datos en formato JSON e acceder aos mesmos a través dun navegador via HTTP, 
funcionando como unha API Rest.

    Permite gran cantidade de funcionalidades como servir aplicacións 
directamente desde CouchDB así como un sistema de replicación incremental e de 
detección de conflictos.

  \end{description}

  \section{Estándares de comunicación}

  \begin{description}
   \item [JSON] É un formato estándar para o intercambio de datos e que pola súa 
simplicidade estase a impoñer como formato habitual por exemplo, para a comunicación con 
APIs Rest e debido a súa similitude coa definición de obxectos en Javascript, permite que 
sexa tremendamente sinxelo traballar con él dende esta linguaxe.
  \end{description}

  \section{Repositorios de código}

  \begin{description}
   \item [Github] Github é un repositorio de código que se está a convertir no lugar máis 
importante de publicación de aplicacións Software Libre e que permite aloxar proxectos 
como o presente, de forma totalmente gratuita.
  Cómpre destacar que esta é a única ferramenta utilizada para o desenvolvemento do 
proxecto que non é software libre pero si proporciona unha visibilidade de cara a 
comunidade de gran importancia neste tipo de proxectos.
  \end{description}

  \section{Ferramentas de xestión}

  \begin{description}
   \item [Git] É un sistema de control de versións software libre de gran 
potencia e 
utilizado en millóns de proxectos que aporta unha versatilidade enorme ao ser 
distribuida, permitindo traballar incluso de forma offline.
   \item [Gulp] Un sistema que permite a automatización de tarefas durante o 
desenvolvemento de aplicacións como por exemplo compilar automáticamente o código 
Javascript escrito na súa última versión á versión máis antiga para que poida ser 
executada por calquera navegador web.
    \item [Babel] É un compilador de Javascript que permite a traducir código 
fonte escrito no estándar ECMAScript 6 a ECMAScript 2015, soportado pola gran 
maioría de navegadores.
    \item [Browserify] É unha ferramenta que permite escribir os módulos da 
aplicación como se fosen módulos para unha aplicación en Node.js e que os 
compila para poder ser utilizados no navegador web. 
   \item [Redmine] É unha ferramenta de xestión de proxectos flexible, multiplataforma e 
software libre con diversos plugins para facilitar a planificación de iteracións e 
traballar con metodoloxías áxiles de desenvolvemento.
   \item [Travis CI] É unha ferramenta de integración continua que permite automatizar a 
execución de tests ou o despregamento automático de código. Ademáis dispón dunha 
integración con Github polo que resulta moi sinxelo automatizar estas tarefas.
   \item [Docker] É un sistema que permite empaquetar e despregar de xeito 
sinxelo aplicacións coas súas dependencias en unidades estándar chamadas 
contenedores, abstraendo e automatizando a virtualización da plataforma na que 
correrá a aplicación.
   \item [Atom] Un editor de texto software libre deseñado inicialmente por Github de 
gran potencia e extensibilidade gracias a un sinxelo sistema de plugins. Ademáis é un 
editor impulsado por Facebook (creadores de ReactJS) para facilitar o traballo con esta 
tecnoloxía.

  \end{description}

  \section{Ferramentas documentáis}

  \begin{description}
   \item [LaTeX] Un sistema para a composición de documentos que inclúe todo tipo de 
funcionalidades para a edición de textos científicos ou técnicos, moi adecuado para este 
proxecto e que xenera documentos de gran calidade.
   \item [Dia] É unha aplicación para a creación de diagramas entre os que se atopan os 
diagramas UML e que permite a exportación dos mesmos a imáxenes vectoriais.
  \end{description}

  
  
  