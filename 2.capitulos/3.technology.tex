\chapter{Fundamentos tecnolóxicos}
\minitoc
% \label{chap:Fundamentostecnoloxicos}
% \vspace{0.5cm}

%%%%%%%%%%%%%%%%%%%%%%%%%%%%%%%%%%%%%%%%%%%%%%%%%%%%%%%%%%%%%%%%%%%%%%%%%%%%%%%%
% Objetivo:                        %
%%%%%%%%%%%%%%%%%%%%%%%%%%%%%%%%%%%%%%%%%%%%%%%%%%%%%%%%%%%%%%%%%%%%%%%%%%%%%%%%

  \lettrine{N}{este} capítulo móstraranse as diversas tecnoloxías que foron 
empregadas durante o desenvolvemento do proxecto así como ferramentas de 
xestión, bases de datos, repositorios de código e ferramentas documentais, 
todas, tecnoloxías Software Libre.

  \section{Linguaxes e frameworks empregados}

  \begin{description}
   \item [ReactJS] 
\href{https://facebook.github.io/react/}{facebook.github.io/react} \\ React é 
unha librería de Javascript para a 
creación de Single Page Aplications (SPAs), permitindo crear aplicacións 
completas que se executen no navegador de forma sinxela a través de diversos 
compoñentes que se agrupan e que permiten crear aplicacións multiplataforma.

   \item [Reflux] \href{https://github.com/reflux/refluxjs}{
github.com/reflux/refluxjs}\\ É unha implementación da 
arquitectura Flux que explicaremos no Capítulo~\ref{sec:design:react} e que 
permite un fluxo de datos unidireccional, en lugar do 
bidireccional que é habitual nas aplicacións web.
    Permite tamén a creación de Stores nas que se mantén o estado das aplicacións e que 
permite compartir dito estado entra os diversos compoñentes da aplicación.

   \item [Jest] 
\href{https://facebook.github.io/jest/}{facebook.github.io/jest}\\ É unha 
ferramente sinxela creada sobre o 
framework de testing para Jasvascript, Jasmine, que facilita a utilización de 
mocks e a creación de tests unitarios.

   \item [React Intl (i18n)]\href{http://formatjs.io/}{formatjs.io}\\ React 
Intl é unha 
ferramenta para facilitar a internacionalización de aplicacións Javascript e 
concretamente as 
aplicacións baseadas en React.

   \item [React 
Router]\href{https://github.com/reactjs/react-router}{
github.com/reactjs/react-router}\\ Unha 
librería para o enrutado de aplicacións baseadas en ReactJS 
proveendo unha API sinxela con funcionalidades de gran potencia como a carga preguiceira 
de código ou o enrutado dinámico.

   \item [Material UI] 
\href{http://www.material-ui.com/}{material-ui.com}\\ Un conxunto de 
compoñentes para React que implementan o Material 
Design impulsado por Google, unha nova linguaxe visual baseada na representación en 3D 
dos obxectos que non deben intersecarse se non que a través de sombras para simular 
diferentes profundidades, os obxectos debe superpoñerse uns sobre os outros.

  \end{description}

  \section{Bases de datos}

  \begin{description}

     \item [CouchDB] \href{http://couchdb.apache.org/}{couchdb.apache.org} É 
unha base de datos 
pensada para web que permite almacenar 
os datos en formato JSON e acceder aos mesmos a través dun navegador via HTTP, 
funcionando como unha API REST (Representational state 
transfer).

  Permite múltiples funcionalidades pouco habituais entre os sistemas de 
xestión de bases de datos como servir aplicacións directamente dende 
CouchDB así como un sistema de replicación incremental e de detección de 
conflictos.

     \item [PouchDB] \href{https://pouchdb.com/}{pouchdb.com}\\ Unha base de 
datos NoSQL 
baseada en Javascript e inspirada 
en CouchDB, pensada para facilitar o funcionamento de aplicacións web de forma 
offline.

PouchDB permite almacenar os datos localmente no navegador web cando non 
hay conexión a internet e sincronizar de forma sinxela ditos datos en remoto con 
CouchDB e outros servidores compatibles.%

  \end{description}

  \section{Estándares de comunicación}

  \begin{description}
   \item [JSON] \href{http://couchdb.apache.org/}{couchdb.apache.org}\\ É un 
formato estándar para o 
intercambio de datos e que pola súa 
simplicidade estase a impoñer como formato habitual por exemplo, para a comunicación con 
APIs Rest e debido a súa similitude coa definición de obxectos en Javascript, permite que 
sexa tremendamente sinxelo traballar con él dende esta linguaxe.
  \end{description}

  \section{Repositorios de código}

  \begin{description}
   \item [Gerrit] 
\href{https://www.gerritcodereview.com/}{gerritcodereview.com}\\ Gerrit é 
un repositorio de código baseado no sistema de control de 
versións Git e centrado en proveer un xeito sinxelo de realizar revisións de 
código dende unha plataforma web. Foi utilizado durante o desenvolvemento do 
prototipo da aplicación pero finalmente decidíuse trasladar o código a GitHub 
para facilitar a colaboración de outros posibles desenvolvedores.
   \item [GitHub] \href{http://github.com/}{github.com}\\ GitHub é un 
repositorio de 
código que se está a convertir no lugar máis 
importante de publicación de aplicacións Software Libre e que permite aloxar proxectos 
como o presente, de forma totalmente gratuita.
  Cómpre destacar que esta é a única ferramenta utilizada para o desenvolvemento do 
proxecto que non é software libre pero si proporciona unha visibilidade de cara a 
comunidade de gran importancia neste tipo de proxectos.
  \end{description}

  \section{Ferramentas de xestión}

  \begin{description}
   \item [Git] \href{https://git-scm.com/}{git-scm.com}\\ É un sistema de 
control de 
versións software libre con grandes funcionalidades e que é utilizado en 
millóns de proxectos. Aporta unha versatilidade enorme ao ser distribuido, 
permitindo traballar incluso de forma offline.
   \item [Gulp] \href{http://gulpjs.com/}{gulpjs.com}\\ Un sistema que permite 
a 
automatización de tarefas durante o desenvolvemento de aplicacións como por 
exemplo compilar automáticamente o código Javascript escrito na súa última 
versión á versión máis antiga para que poida ser executada por calquera 
navegador web.
    \item [Babel] \href{https://babeljs.io/}{babeljs.io}\\ É un compilador de 
Javascript que permite a traducir código fonte escrito no estándar ECMAScript 6 
a ECMAScript 2015, soportado pola gran maioría de navegadores.
    \item [Browserify] \href{http://browserify.org/}{browserify.org}\\ É unha 
ferramenta 
que permite escribir os módulos da aplicación como se fosen módulos para unha 
aplicación escrita en Node.js e que os compila para poder ser utilizados no 
navegador web.
   \item [Redmine] \href{http://www.redmine.org/}{redmine.org}\\ É unha 
ferramenta de 
xestión de proxectos flexible, multiplataforma e software libre con diversos 
plugins para facilitar a planificación de iteracións e traballar con 
metodoloxías áxiles de desenvolvemento.
   \item [Travis CI] \href{https://travis-ci.org/}{travis-ci.org}\\ É unha 
ferramenta de 
integración continua que permite automatizar a execución de tests ou o 
despregamento automático de código. Ademáis dispón dunha integración con Github 
polo que resulta moi sinxelo automatizar estas tarefas.
   \item [Docker] \href{https://www.docker.com/}{docker.com}\\ É un sistema 
que permite 
empaquetar e despregar de xeito sinxelo aplicacións coas súas dependencias en 
unidades estándar chamadas contenedores, abstraendo e automatizando a 
virtualización da plataforma na que correrá a aplicación.
   \item [Atom] \href{https://atom.io/}{atom.io}\\ Un editor de texto software 
libre 
deseñado inicialmente por GitHub e centrado na súa extensibilidade gracias a un 
sinxelo sistema de plugins. Ademais, no ecosistema de ReactJS, existen 
compoñeentes para Atom co obxectivo de facilitar a edición de aplicacións React, 
elaboradas polos mesmos impulsores da propia librería.
   \item [Apache Cordova] 
\href{https://cordova.apache.org/}{cordova.apache.org}\\ Unha 
ferramenta de desenvolvemento que permite usar tecnoloxías web estándar (HTML, 
CSS3 e Javascript) para crear aplicacións móbiles múltiplataforma.

  \end{description}

  \section{Ferramentas documentais}

  \begin{description}
   \item [LaTeX] \href{https://www.latex-project.org/}{latex-project.org}\\ 
Un sistema para a 
composición de documentos que inclúe todo tipo de funcionalidades para a edición 
de textos científicos ou técnicos, moi adecuado para este proxecto e que xenera 
documentos de xeito sinxelo e automáticamente estruturados.
   \item [Dia] \href{http://dia-installer.de/}{dia-installer.de}\\ É unha 
aplicación para a 
creación de diagramas entre os que se atopan os diagramas UML e que permite a 
exportación dos mesmos a imaxes vectoriais.
  \end{description}
