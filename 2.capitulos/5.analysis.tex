\chapter{Análise de requisitos globais}
\minitoc
% \label{chap:Analisederequisitosglobais}
% \vspace{0.5cm}

%%%%%%%%%%%%%%%%%%%%%%%%%%%%%%%%%%%%%%%%%%%%%%%%%%%%%%%%%%%%%%%%%%%%%%%%%%%%%%%%
% Objetivo:                        %
%%%%%%%%%%%%%%%%%%%%%%%%%%%%%%%%%%%%%%%%%%%%%%%%%%%%%%%%%%%%%%%%%%%%%%%%%%%%%%%%

  \lettrine{N}{este} capítulo exporemos o proceso de análise de requisitos 
para o desenvolvemento do proxecto, explicando as diversas visitas que se 
realizaron a múltiples federacións.

  Durante meses traballouse da man de varias de estas federacións e asociacións 
deportivas tratando de comprender, non só as necesidades reais dos clientes se 
non tamén traballando na usabilidade da aplicación da súa man.

  Comezaremos vendo as suxerencias recibidas das diversas federacións e 
finalmente expoñeremos os requisitos que finalmente se decidíu engadir ao 
proxecto.

  \section{Consultas a xestores de federacións}
  Para a realización deste apartado decidiuse consultar con diversas 
asociacións deportivas e federacións das que obter suxerencias e peticións 
acerca das necesidades que actualmente están a demandar, co fin de obter certas 
funcionalidades a implementar e incluso a priorización segundo as súas
necesidades máis urxentes.

Na realización deste apartado contouse coa colaboración das asociacións e 
federacións deportivas que se detallan a continuación.

  Tras varias reunións con eles, obtívose unha lista de requerimentos e 
suxerencias respecto das súas necesidades, que están detalladas 
na Sección~\ref{sec:analisis:obtido}, e dos que finalmente foron destilados os 
requisitos finais, que son presentados como 
Sección~\ref{sec:analisis:requisitos} de este capítulo.

  \begin{description}

  \item [Asociación de peñas de fútbol de A Coruña].
  É a asociación máis interesada polo proxecto e coa que se leva colaborando 
dende o primeiro momento, aportando suxerencias e incluso novos colaboradores 
para poder desenvolver un producto de calidade.

  Realizáronse ata 5 visitas á federación co fin de mostrarlles a evolución do 
proxecto, comprobar a usabilidade da aplicación e o estado das funcionalidades.

  \item [UPOFU].
  A Asociación de peñas ten boa relación coa UPOFU polo que nos facilitóu o seu 
contacto e ofrecéronse da mesma maneira a colaborar co proxecto, interesados 
tamén en incorporalo na súa xestión.

  \item [Torneo VACmatch].
  Organizóuse un torneo de fútbol sala co fin de testear o primeiro prototipo 
do proxecto con usuarios reais, tanto árbitros como xogadores e no que se 
comprobou as dificultades dos usuarios e se verificou a súa necesidade de 
dispor deste tipo de ferramentas.

  \item [Outras].
  Tamén se realizaron visitas a outras federacións incluso de outros deportes 
como o voleibol para comprobar os seus problemas na xestión e verificar a 
importancia de que o proxecto sexa facilmente adaptable a outros deportes.

  \end{description}

  \section{Peticións obtidas}
  \label{sec:analisis:obtido}
    \begin{description}
     \item [Cubrir acta en tempo real]. A aplicación móbil debe permitir cubrir 
as actas e actualizar os resultados en tempo real co fin de manter a web da 
federación actualizada en todo momento.

     \item [Permisos]. Débese dispor dun sistema de permisos para diferenciar a 
árbitros e outros xestores da competición.

     \item [Sincronización]. A aplicación móbil debe sincronizar os datos coa 
plataforma central onde se atopa o sistema de xestión da federación e a súa web.

     \item [Persoas convocadas]. É preciso poder dispor de todas as persoas 
inscritas nun equipo e poder indicar de xeito sinxelo si esas persoas están ou 
non no encontro.

     \item [Eventos]. A aplicación debe poder crear novos eventos, borralos e 
mostralos de xeito sinxelo e ao mesmo tempo de xeito xenérico que permita 
integrar calquera deporte.

     \item [Motivación dun evento]. Debe poderse incluir en certos eventos un 
motivo polo que se creou ese evento, dispoñendo dunha lista de motivos por 
defecto e incluso permitindo ao xestor da federación, engadir novos motivos 
personalizados para a súa federación.

     \item [Editar dorsal dun xogador]. Xa que en moitas competicións un 
xogador pode xogar cada partido con un dorsal diferente, debe poder cambiarse o 
dorsal por defecto dende a aplicación móbil.

     \item [Persoa con varios roles]. Debe terse en conta a posibilidade de que 
unha persoa poida ter varios roles, tanto de xogador como de entrenador dentro 
de un equipo.

    \end{description}

  \section{Requisitos finais}
  \label{sec:analisis:requisitos}
  \subsection{Usuarios}

    \begin{itemize}

    \item \textbf{Facer login e logout}.
    A aplicación móbil debe permitir iniciar e pechar sesión para os árbitros.

    \item \textbf{Permisos para edición de actas}.
    A aplicación de xestión disporá de permisos diferenciados para editar as 
actas xa que os árbitros únicamente poden editar as actas que teñen asignadas.

    \end{itemize}

  \subsection{Listar actas}

    \begin{itemize}

    \item \textbf{Visualizar próximas actas a cubrir dun árbitro}.
    Mostrar a lista de próximas actas que ten para cubrir un árbitro, mostrando 
o lugar e a data do mesmo para facilitar o seu traballo.

    \item \textbf{Visualizar actas cubertas dun árbitro}.
    Debe mostrar as actas cubertas anteriormente e todos os seus datos.

    \item \textbf{Actualización automática de actas descargadas ante 
modificacións}.
    As actas deben actualizarse de forma automática na aplicación do árbitro 
unha vez o xestor da federación realiza a asignación dun partido a un colexiado.

    \end{itemize}

  \subsection{Visualizar actas}

    \begin{itemize}

    \item \textbf{Listar o personal e xogadores dun equipo}.
    Móstrase o personal e os xogadores do equipo na aplicación móbil.

    \item \textbf{Visualizar datos xerais dun acta}.
    Débese mostrar do xeito máis simplificado posible os datos xerais da acta 
nunha pantalla inicial para facilitar que sexa cuberta interactivamente durante 
o desenvolvemento do encontro.

    \item \textbf{Visualizar eventos dun acta}.
    Permitirase visualizar os eventos ordenados cronolóxicamente para facilitar 
a súa consulta.

    \end{itemize}

  \subsection{Xeración de actas offline}
  A aplicación debe permitir a creación de actas de forma offline xa que pódese 
dar o caso de que a aplicación móbil non actualice as novas actas e o árbitro 
se vexa na obriga de crear unha acta de forma manual.

  \subsection{Modificación de actas}

    \begin{itemize}

    \item \textbf{Modificación de propiedades da acta}.
    Débese permitir modificar propiedades da acta tales como a localización 
do encontro ou a data do mesmo.

    \item \textbf{Convocar un xogador ou entrenador}.
    A aplicación móbil permitirá indicar qué personal dos equipos están 
presentes no encontro así como editar certos datos dos mesmos como o seu dorsal.

    \item \textbf{Engadir un xogador que non está no equipo}.
    Pode darse o caso de que a un xogador débeselle permitir xogar un encontro 
aínda que non fose dado de alta na federación correspondente polo que é preciso 
poder engadir novos xogadores.

    \item \textbf{Editar datos de persoal creado}.
    Débese permitir editar certos datos dun xogador que foi creado dende 
a aplicación móbil como o nome, o dorsal ou o equipo o que pertencen.

    \item \textbf{Poder engadir motivos dun evento xerado}.
    A federación debe poder engadir novos motivos personalizados para poder 
engadir a un evento dende a aplicación de xestión.

    \item \textbf{Cambiar de parte}.
    A aplicación móbil debe permitir cambiar de parte no encontro.

    \item \textbf{Modificar o tempo}.
    A aplicación móbil disporá dun cronómetro que permita seguir o tempo do 
encontro así como permitirá modificalo manualmente por si hai algún desaxuste 
durante o encontro.

    \item \textbf{Engadir observacións na acta}.
    O árbitro debe poder engadir observacións as actas dos encontros.

    \item \textbf{Asinar a acta}.
    Tanto persoal do equipo como árbitros deben poder asinar as actas con un 
código PIN do que disporá cada un.

    \item \textbf{Engadir eventos deportivos}.
    A aplicación debe facilitar a adaptación de novos deportes e a posibilidade 
de engadir de xeito sinxelo novos eventos.

    \end{itemize}


