\chapter{Deseño e implementación}
\minitoc
% \label{chap:Desenoeimplementacion}
% \vspace{0.5cm}

%%%%%%%%%%%%%%%%%%%%%%%%%%%%%%%%%%%%%%%%%%%%%%%%%%%%%%%%%%%%%%%%%%%%%%%%%%%%%%%%
% Objetivo:                        %
%%%%%%%%%%%%%%%%%%%%%%%%%%%%%%%%%%%%%%%%%%%%%%%%%%%%%%%%%%%%%%%%%%%%%%%%%%%%%%%%

  \lettrine{E}{n} este capítulo...

  \santiagosays{Deseño e implementación son o mesmo capítulo.
    Dependendo de como o queiras enfocar, pode ser o correcto, ou
    podes dividilo en dous. Do mesmo xeito, fixeches realtivo fincapé
    no TDD, e en ningunha parte da implementaión falas do esting. En
    particular, non nomeas como facer testing máis aló das probas
    unitarias automáticas, cousas como cales son os test de aceptación
    por parte das federacións e usuarios para co sistema, ou como se
    axustan aos estándares de calidade cousas tipo Cordova.

    Se o testing é importante, faino ver cun capítulo no que se conten
    cousas. Se pola contra \emph{non hai moito que contar}, merece
    polo menos unha sección aquí. Tamén cómpre dividir o capítulo
    entre as decisións relativas ao deseño e á implementación, se se
    vai a plantexar así: por un lado, \textbf{é unha decisión de
      deseño} empregar a arquitectura Flux, mentres que programar as
    vistas en Javascript, ou a xestión asíncrona das accións usando
    Reflux particularmente, son detalles de implementación.

    Do mesmo xeito, é unha decisión de deseño desenvolver o proxecto
    empregando tecnoloxías web, e é un dealle de implementación
    utilizar Apache Cordova para facilitar a experiencia a usuarios
    móbiles proporcionando unha interface (un pelín) máis amigable.

    Tamén é unha decisión de deseño empregar unha DB non relacional
    para gardar os datos. Sen embargo, é un requisito funcional que
    funcionase offline. Por iso se empregou PouchDB no cliente
    (detalle de implementación) para facer iso posible sen ter que
    implementar nós o ``caché'' de CouchDB.

    Que a signtatura se faga con PIN é un detalle de implementación,
    xa que se pode cambiar, e o deseño segue a ser o mesmo: que o
    usuario/capitan/blah ten a capacidade de firmar e estar dacordo ou
    non co reflictido na acta.}

  \section{ReactJS e Flux}
    \subsection{Intro}
    \subsection{Elementos básicos}
    Componentes, actions, stores...
    \subsection{Fluxo das aplicacións}
    Explicación práctica do fluxo
    \subsection{App híbrida con Apache Cordova}

    \subsection{Estructura do código}

  \section{PouchDB e funcionamento offline}
  Creación de actas offline e de xogadores

  \section{Interface gráfica e usabilidade}
  Estructura a nivel de deseño da aplicación. Reutilización de compoñentes xenéricos.

    \subsection{Elementos comúns}

      \subsubsection{Menú lateral esquerdo}
      Cómo se estructura e se engaden novos elementos.

      \subsubsection{Enlaces do menu superior dereito}
      Cómo se estructura e se engaden novos.

      \subsubsection{Información e axustes}

      \subsubsection{SnackBar}
      Para xestión de erros

    \subsection{Listado de actas}


    \subsection{Acta}

      \subsubsection{Convocar xogadores}

      \subsubsection{Inicio e fin do partido}
      Non se poden engadir eventos ata o de inicio pero si despois do fin...

      \subsubsection{Cronómetro}

      \subsubsection{Tipos de eventos}
    Control o de deporte
      \paragraph{Eventos de control}
      \paragraph{Eventos de deporte}
        \subparagraph{Eventos de puntuación (Score events)}
        Que actualizan tamén a acta e o estado da aplicación.

    \subsection{Finalización do encontro}

      \subsubsection{Incidencias}

      \subsubsection{Sinatura con PIN}

  \section{Multideporte}
  Almacénase nunha Store e compártese xa que ten a lóxica das accións que dependen do 
  deporte como os eventos que poden utilizarse.

  \section{Redeseño da DB}
  API de VACmatch ten unha BD relacional e tívose que realizar un redeseño para a DB no 
móbil.

  \section{I18n}

  \section{Injección de dependencias}
  Dependencias circulares entre servicios.

  \section{Integración continua}
  Jest y Travis CI

