\chapter{Deseño e implementación}
\minitoc
% \label{chap:Desenoeimplementacion}
% \vspace{0.5cm}

%%%%%%%%%%%%%%%%%%%%%%%%%%%%%%%%%%%%%%%%%%%%%%%%%%%%%%%%%%%%%%%%%%%%%%%%%%%%%%%%
% Objetivo:                        %
%%%%%%%%%%%%%%%%%%%%%%%%%%%%%%%%%%%%%%%%%%%%%%%%%%%%%%%%%%%%%%%%%%%%%%%%%%%%%%%%

  \lettrine{E}{n} este capítulo...

  \section{ReactJS e Flux}
    \subsection{Intro}
    \subsection{Elementos básicos}
    Componentes, actions, stores...
    \subsection{Fluxo das aplicacións}
    Explicación práctica do fluxo
    \subsection{App híbrida con Apache Cordova}

    \subsection{Estructura do código}

  \section{PouchDB e funcionamento offline}
  Creación de actas offline e de xogadores

  \section{Interface gráfica e usabilidade}
  Estructura a nivel de deseño da aplicación. Reutilización de compoñentes xenéricos.

    \subsection{Elementos comúns}

      \subsubsection{Menú lateral esquerdo}
      Cómo se estructura e se engaden novos elementos.

      \subsubsection{Enlaces do menu superior dereito}
      Cómo se estructura e se engaden novos.

      \subsubsection{Información e axustes}

      \subsubsection{SnackBar}
      Para xestión de erros

    \subsection{Listado de actas}


    \subsection{Acta}

      \subsubsection{Convocar xogadores}

      \subsubsection{Inicio e fin do partido}
      Non se poden engadir eventos ata o de inicio pero si despois do fin...

      \subsubsection{Cronómetro}

      \subsubsection{Tipos de eventos}
    Control o de deporte
      \paragraph{Eventos de control}
      \paragraph{Eventos de deporte}
        \subparagraph{Eventos de puntuación (Score events)}
        Que actualizan tamén a acta e o estado da aplicación.

    \subsection{Finalización do encontro}

      \subsubsection{Incidencias}

      \subsubsection{Sinatura con PIN}

  \section{Multideporte}
  Almacénase nunha Store e compártese xa que ten a lóxica das accións que dependen do 
  deporte como os eventos que poden utilizarse.

  \section{Redeseño da DB}
  API de VACmatch ten unha BD relacional e tívose que realizar un redeseño para a DB no 
móbil.

  \section{I18n}

  \section{Injección de dependencias}
  Dependencias circulares entre servicios.

  \section{Integración continua}
  Jest y Travis CI

