\chapter{Deseño e implementación}
\minitoc
\label{chap:Desenoeimplementacion}
\vspace{0.5cm}

%%%%%%%%%%%%%%%%%%%%%%%%%%%%%%%%%%%%%%%%%%%%%%%%%%%%%%%%%%%%%%%%%%%%%%%%%%%%%%%%
% Objetivo:                        %
%%%%%%%%%%%%%%%%%%%%%%%%%%%%%%%%%%%%%%%%%%%%%%%%%%%%%%%%%%%%%%%%%%%%%%%%%%%%%%%%

  \lettrine{E}{n} este capítulo...

  \section{ReactJS e Flux}
    \subsection{Intro}
    \subsection{Elementos básicos}
    Componentes, actions, stores...
    \subsection{Fluxo das aplicacións}
    Explicación práctica do fluxo
    \subsection{Estructura do código}

  \section{Interface gráfica}
  Estructura a nivel de deseño da aplicación. Reutilización de compoñentes xenéricos.

  \section{Unhosted APP}
  Creación de actas offline e de xogadores

  \section{Inicio e fin do partido}
  Non se poden engadir eventos ata o de inicio pero si despois do fin...
  
  \section{Deporte}
  Almacénase nunha Store e compártese xa que ten a lóxica das accións que dependen do 
deporte como os eventos que poden utilizarse.
  
  \section{Redeseño da DB}
  API de VACmatch ten unha BD relacional e tívose que realizar un redeseño para a DB no 
móbil.
  
  \section{Enlaces do menu superior dereito}
  Cómo se estructura e se engaden novos.

  \section{Tipos de eventos}
  Control o de deporte
    \subsection{Eventos de control}
    \subsection{Eventos de deporte}
      \subsubsection{Enventos de puntucación (Score events)}
      Que actualizan tamén a acta e o estado da aplicación.

  \section{Resolución de conflictos???}
  Aplicación mobil offline con conflictos a resolver

  \section{Integración con VACmatch}