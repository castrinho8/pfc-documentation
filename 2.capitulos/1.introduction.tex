\chapter{Introducción}
\minitoc
% \label{chap:introduccion}

%%%%%%%%%%%%%%%%%%%%%%%%%%%%%%%%%%%%%%%%%%%%%%%%%%%%%%%%%%%%%%%%%%%%%%%%%%%%%%%%
% Objetivo: Exponer de qué vai este proxecto, a súas líñs mestras, obxetivos,   %
%           etc.                                                               %
%%%%%%%%%%%%%%%%%%%%%%%%%%%%%%%%%%%%%%%%%%%%%%%%%%%%%%%%%%%%%%%%%%%%%%%%%%%%%%%%

 \lettrine{N}{este} capítulo trataranse os aspectos básicos para comprender o proxecto 
así como os motivos que levaron ao seu desenvolvemento e a estrutura da presente 
memoria.

  Falaremos do estado do deporte na actualidade e da xestión deportiva en 
concreto, co fin de mostrar a necesidade de impulsar un proxecto dentro de este 
campo para, posteriormente introducir a iniciativa dentro da que este
proxecto xurdíu e rematar con un resumo da problemática que resolve.

 
  \section{O deporte amateur e o avance tecnolóxico}
    Actualmente o deporte é fundamental na vida das persoas, durante os últimos anos o 
número de españois que realizan algunha actividade física medrou enormemente así como o 
número de competicións amateur, que permiten a estos deportistas, competir por un custe 
moito máis asequible que as federacións oficiais.

    Se embargo, este crecemento do número de deportistas non veu acompañado tamén dunha 
renovación tecnolóxica das competicións polo que gran parte dos seus xestores seguen
a invertir un tempo elevado nas súas competicións e non teñen apenas 
relación dixital coas persoas que compiten nas mesma.

    \section{A problemática}
    Actualmente os organizadores de competicións deben realizar unha serie de tarefas que 
se describen a continuación e que na súa meirande parte, realizan de forma manual ou 
axudados de follas de cálculo, ao non dispor das ferramentas tecnolóxicas axeitadas a un 
prezo accesible.

    \begin{description}
     \item [Inscricións] Na maior parte das competicións, os xogadores seguen a ter que 
levar cuberta a súa ficha cos seus datos persoais en papel, fotocopia do DNI, fotografía, 
etc para que a federación garde eses datos nunha folla de cálculo.
     \item [Aplazamentos de partidos] Moitas esíxenlles aos equipos, unha vez postos de 
acordo, enviar unha confirmación en papel, por correo ordinario ou fax.
     \item [Notificacións] Deben avisar aos sancionados, os cambios no calendario, etc 
por correo electrónico cando menos.
     \item [Revisión de sancionados] A federación debe comprobar que un xogador 
sancionado non xogóu un partido que non debía.
     \item [Loxística das actas dos encontros] O árbitro do encontro debe recoller as 
actas na asociación e volver a traelas cubertas despois dos encontros.
     \item [Publicación de resultados] A federación debe recopilar tódolos datos das 
actas para publicalos, ben sexa nunha web ou por email aos participantes.
     \item [Publicación de clasificacións e estadísticas] A federación debe calcular a 
clasificación e recopilar as estadísticas para publicalas posteriormente.
    \end{description}

    O proxecto desenvolto trata de resolver os últimos apartados mencionados no 
punto anterior, \emph{a xestión das actas dos encontros, a súa loxística} e a 
\emph{automatización da publicación de resultados e clasificacións}.

  Para iso decidíuse crear unha aplicación móbil que permita que os árbitros 
xestionen as súas actas directamente dende o seu teléfono móbil ou tableta, 
nunha aplicación multidispositivo baseada en tecnoloxías web, permitindo incluso realizar 
ditas actas sen conexión a internet, algo que hoxe en día ningunha aplicación ofrece no 
mercado nacional.

  Esta aplicación móbil integrarase tamén con un sistema de xestión de 
competicións co fin de que os árbitros poidan cubrir as actas e publicar as 
estadísticas e resultados directamente na web da federación, a través do seu 
sistema de xestión.

    \section{VACmatch}
    VACmatch é unha iniciativa empresarial xurdida na Universidade da Coruña para 
mellorar a xestión de competicións deportivas a través dunha serie de 
aplicacións entre as que se atopa este proxecto.

    A iniciativa recibíu o pasado ano a calificación de \emph{Iniciativa Empresarial de 
Base Tecnolóxica (IEBT)} recoñecendo o seu grao de innovación así como participou en 
diversos programas de apoio a ideas emprendedoras como \emph{Yuzz} 
\footnote{Programa de formación empresarial do Centro Internacional Santander 
Emprendimiento} ou \emph{Telefónica Galicia OpenFuture\_} \footnote{Programa de 
mentorización e formación de negocio de Telefónica con un premio de 2.000 
\euro{}} e durante case un ano, convivíu con 
outras iniciativas empresariais no \emph{Viveiro de empresas da Universidade da 
Coruña}.

    Ano e medio despois do seu comezo decidíuse abandonar o proxecto 
como iniciativa empresarial pero VACmatch continúa como comunidade baseada nun 
proxecto de software libre.

    \section{Resumo do proxecto}
    O proxecto desenvolto componse de dúas partes diferenciadas que permite a 
xestión das actas dos encontros por parte das federacións deportivas.
    
    \subsection{VACmatch Mobile}
    É unha aplicación móbil híbrida realizada con tecnoloxías web co fin de 
poder utilizala en calquera plataforma, tanto a través da web como nun móbil 
Android, IOS, FirefoxOS... e que permitirá aos árbitros das competiciones 
realizar todas as xestións coas actas dos encontros dende o seu teléfono.

  \begin{description}
    \item [Lista de actas]
    Esta aplicación permite que os árbitros poidan dispoñer no seu teléfono 
das actas dos partidos que teñen que dirixir, coa localización e a data dos 
mesmos e que se actualizan de forma automática cando se reasignan ou se 
cancelan.

    \item [Convocatoria de xogadores]
    Unha vez o árbitro chega ao encontro pode seleccionar na aplicación os 
xogadores que asistiron ao mesmo únicamente con un click, introducir a algún 
novo se o desexa ou editar datos como o dorsal dun xogador.

    \item [Xestión de actas]
    Unha vez comezado o encontro, a aplicación permitirá introducir os diversos 
eventos que ocorren no mesmo como infraccións, goles ou tarxetas de forma que é 
moi sinxelo engadir novos deportes e eventos.

    \item [Sinatura de actas]
    Para rematar o encontro, o árbitro poderá engadir comentarios a mesma acta 
e tanto él como os xogador ou persoal dos equipos, poderán asinar a acta con un 
código PIN do que dispón cada un.

    \item [Actas offline]
    O árbitro poderá crear actas incluso aínda que non tivese sincronizados 
todos os datos do partido, permitindo cubrir as actas incluso no peor escenario 
posible.

  \end{description} 

    \subsection{VACmatch Web}
    A aplicación móbil explicada anteriormente atópase nas primeiras fases de 
un proceso de integración con unha aplicación web, a través da cal as 
federacións poden xestionar completamente as súas competiciones, modificar o 
calendario, engadir novos xogadores ou equipos, xestionar arbitraxes, etc.

    Actualmente permítese que a federación cree as actas dos encontros, os 
árbitros as sincronicen nos seus teléfonos e unha vez cubertas, todos os datos 
sexan publicados automáticamente na web da federación a través dun pequeno 
plugin.

    Esto facilita que a federación poida dar o primeiro paso de sustituir as 
actas e fichas físicas por versións dixitais pero deben continuar facendo a 
integración dos datos das actas de xeito manual, xa que ambos sistemas funcionan 
por separado, e polo tanto é preciso comprobar as actas e mover os seus datos 
ao sistema de xestión de VACmatch Web onde se gardan os datos finais e 
verificados pola federación.

    Nun futuro cercano permitirase manter as clasificacións actualizadas en todo 
momento sen apenas intervención humana, e aforrando un enorme traballo na 
revisión das actas.

  \subsection{Estrutura da memoria}
  
  \todo{Engadir estrutura}
  
  