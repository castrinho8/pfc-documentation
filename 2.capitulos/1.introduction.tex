\chapter{Introducción}
\minitoc
\label{chap:introduccion}

%%%%%%%%%%%%%%%%%%%%%%%%%%%%%%%%%%%%%%%%%%%%%%%%%%%%%%%%%%%%%%%%%%%%%%%%%%%%%%%%
% Objetivo: Exponer de qué va este proyecto, sus líneas maestras, objetivos,   %
%           etc.                                                               %
%%%%%%%%%%%%%%%%%%%%%%%%%%%%%%%%%%%%%%%%%%%%%%%%%%%%%%%%%%%%%%%%%%%%%%%%%%%%%%%%

\section{Título del proyecto}

 \lettrine{E}{l} proyecto que en esta memoria se expone lleva por título
 ``\textbf{TÍTULO DEL PROYECTO}''.


 
  \section{O deporte amateur}
    Actualmente o deporte é fundamental na vida das persoas, durante os últimos anos o 
número de españois que realizan algunha actividade física medróu enormemente así como o 
número de competicións amateur que permiten a estos deportistas, competir por un custe 
moito máis asequible que as federacións oficiáis.

    Se embargo, este crecemento do número de deportistas non veu acompañado tamén dunha 
renovación tecnolóxica dos xestores polo que gran parte destas competicións seguen 
a invertir un tempo elevado nas súas competicións e non teñen apenas relación dixital cas 
persoas que compiten nas mesmas.

    \section{Motivación}
    Actualmente os organizadores de competicións deben realizar unha serie de tarefas que 
se describen a continuación e que na súa meirande parte, realizan de forma manual ao non 
dispor das ferramentas tecnolóxicas axeitadas a un prezo accesible.

    \begin{description}
     \item [Inscripcións] Na maior parte das competicións seguen a realizar esto en papel.
     \item [Aplazamentos de partidos] Moitas esixen enviar a confirmación en papel, por 
correo ordinario ou fax.
     \item [Notificacións] Deben avisar aos sancionados por correo electrónico.
     \item [Revisión de sancionados] A federación debe comprobar que un xogador 
sancionado non xogóu un partido que non debía.
     \item [Loxística das actas dos encontros] O árbitro do encontro debe recoller as 
actas na asociación e volver a traelas cubertas despóis dos encontros.
     \item [Publicación de resultados] A federación debe recopilar todos os datos das 
actas para publicalos, ben sexa nunha web ou por email aos participantes.
     \item [Publicación de clasificacións e estadísticas] A federación debe calcular a 
clasificación e recopilar as estadísticas para publicalas posteriormente.
    \end{description}

    Para solucionar todo isto creóuse VACmatch, un sistema de xestión integrado onde 
simplificar todo este traballo, sistema no que se atopa integrado este proxecto, 
\emph{VACmatch Mobile}.

    \section{Definicións}

      \subsection{VACmatch}
    VACmatch é unha plataforma de xestión de competicións deportivas que permite realizar 
todo tipo de trámites coas federacións deportivas de forma electrónica e reduce 
enormemente o traballo que estas deben realizar no seu día a día.

      \subsection{Actas}
    É o lugar onde se almacena a información sobre un encontro deportivo, inclue os 
equipos, os lugares onde se xogou e o resto de estadísticas de cada xogador durante o 
partido.

      \subsection{Fichas}
    Unha ficha é un documento con fotografía incluida que identifica a un xogador que 
compite nunha competición e que debe levar a tódolos encontros para poder disputar os 
partidos.

      \subsection{Xestor da competición}
      Persoa encargada da xestión do calendario, da recepción das actas dos encontros, da 
súa revisión, da súa publicación e, en xeral, da xestión dunha competición.
    
      \subsection{Árbitro}
      Persoa que se encarga de velar polo complimento do regulamento dun deporte durante 
un encontro e así mesmo debe tomar nota na acta, das estadísticas e dos diversos eventos 
que ocurren nun encontro.

    \section{Problemática que resolve VACmatch Mobile}
    A aplicación desenvolta trata de resolver os últimos apartados mencionados no 
apartado anterior, a xestión das actas dos encontros e a automatización da publicación de 
resultados e clasificacións.

    Actualmente o proceso de xestión dun encontro é o seguinte:

    \begin{enumerate}
     \item A federación crea un calendario de encontros que publica na súa web.
     \item Un árbitro recolle un \textbf{acta} e leva ao campo ou a pista na que se 
disputa o encontro que debe arbitrar.
     \item Cada xogador leva a súa \textbf{ficha identificativa} ao encontro.
     \item O árbitro cubre o \textbf{acta} cos datos de tódalas \textbf{fichas} 
dos jugadores.
     \item Durante o encontro, o árbitro de mesa ou o 4º árbitro enche o \textbf{acta} 
manualmente, cubrindo as estadísticas do encontro.
     \item O \textbf{acta} asínase polo árbitro e un representante de cada equipo.
     \item Cada clube guarda unha copia do \textbf{acta} e o árbitro transalada a súa ata 
a federación.
     \item A federación revisa o \textbf{acta}, comprobando que os xogadores que xogaron 
non estaban sancionados, se pertence ao equipo e copiando tódolos datos recollidos, a 
aplicación de xestión ou a fólla de cálculo da que dispoñan.
    \end{enumerate}
  
  É por isto polo que se decidíu crear unha aplicación mobil que permita que os árbitros 
xestionen as súas actas de forma eléctrónica directamente desde o séu teléfono mobil, 
nunha aplicación multidispositivo baseada en tecnoloxías web, permitindo incluso realizar 
ditas actas sen conexión a internet, algo que hoxe en día ningunha aplicación ofrece no 
mercado nacional.

  \subsection{Estrutura da memoria}
