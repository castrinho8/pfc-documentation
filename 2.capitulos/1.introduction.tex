\chapter{Introducción}
\minitoc
% \label{chap:introduccion}

%%%%%%%%%%%%%%%%%%%%%%%%%%%%%%%%%%%%%%%%%%%%%%%%%%%%%%%%%%%%%%%%%%%%%%%%%%%%%%%%
% Objetivo: Exponer de qué vai este proxecto, a súas líñs mestras, obxetivos,   %
%           etc.                                                               %
%%%%%%%%%%%%%%%%%%%%%%%%%%%%%%%%%%%%%%%%%%%%%%%%%%%%%%%%%%%%%%%%%%%%%%%%%%%%%%%%

 \lettrine{N}{este} capítulo trataranse os aspectos básicos para comprender o proxecto 
así como os motivos que levaron ao seu desenvolvemento e a estrutura da presente memoria.

\santiagosays{Esto necesita máis texto de introducción. O típico é que
  introduzcas cada unha das seccións e de que vas a falar nelas, con
  referencias cara adiante. (En plan, ``Na Sección 1.1 falaremos de blah'')}
 
  \section{O deporte amateur e o avance tecnolóxico}
  \santiagosays{Eu fusionaría esta sección coa seguinte: estás
    contando os precedentes que levan a montar o proxecto. Ao mellor
    pode ter sentido, unha vez fusionadas, introducir distintas
    subseccións dentro da problemática.}

    Actualmente o deporte é fundamental na vida das persoas, durante os últimos anos o 
número de españois que realizan algunha actividade física medrou enormemente así como o 
número de competicións amateur, que permiten a estos deportistas, competir por un custe 
moito máis asequible que as federacións oficiáis.

    Se embargo, este crecemento do número de deportistas non veu acompañado tamén dunha 
renovación tecnolóxica das competicións polo que gran parte dos seus xestores seguen
a invertir un tempo elevado nas súas competicións e non teñen apenas relación dixital cas 
persoas que compiten nas mesmas.

    \section{A problemática}
    Actualmente os organizadores de competicións deben realizar unha serie de tarefas que 
se describen a continuación e que na súa meirande parte, realizan de forma manual ou 
axudados de follas de cálculo, ao non dispor das ferramentas tecnolóxicas axeitadas a un 
prezo accesible.

    \begin{description}
     \item [Inscricións] Na maior parte das competicións, os xogadores seguen a ter que 
levar cuberta a súa ficha cos seus datos persoais en papel, fotocopia do DNI, fotografía, 
etc para que a federación garde eses datos nunha folla de cálculo.
     \item [Aplazamentos de partidos] Moitas esíxenlles aos equipos, unha vez postos de 
acordo, enviar unha confirmación en papel, por correo ordinario ou fax.
     \item [Notificacións] Deben avisar aos sancionados, os cambios no calendario, etc 
por correo electrónico.
     \item [Revisión de sancionados] A federación debe comprobar que un xogador 
sancionado non xogóu un partido que non debía.
     \item [Loxística das actas dos encontros] O árbitro do encontro debe recoller as 
actas na asociación e volver a traelas cubertas despóis dos encontros.
     \item [Publicación de resultados] A federación debe recopilar tódolos datos das 
actas para publicalos, ben sexa nunha web ou por email aos participantes.
     \item [Publicación de clasificacións e estadísticas] A federación debe calcular a 
clasificación e recopilar as estadísticas para publicalas posteriormente.
    \end{description}

    O proxecto desenvolto trata de resolver os últimos apartados mencionados no 
punto anterior, \emph{a xestión das actas dos encontros, a súa loxística} e a 
\emph{automatización da publicación de resultados e clasificacións}.

    Para comprender o traballo que lles supón aos xestores de 
federacións é interesante ver cómo se realiza actualmente o proceso de xestión 
das actas dun encontro:
%
\santiagosays[]{\scriptsize Ao mellor esto pode ser unha subsección tamén, como digo; se tes suficiente texto co que introducilo antes de dividir as cousas en subseccións (pensa que ter subseccións implica ter máis texto)}

    \begin{enumerate}
     \item A federación crea un calendario de encontros que publica na súa web.
     \item Un árbitro recolle un \textbf{acta} no local da federación e leva dita acta ao 
campo ou a pista na que se disputa o encontro que debe arbitrar.
     \item Cada xogador leva a súa \textbf{ficha identificativa} ao encontro.
     \item O árbitro cubre o \textbf{acta} cos datos de tódalas \textbf{fichas} 
dos xogadores.
     \item Durante o encontro, o árbitro de mesa ou o 4º árbitro enche o \textbf{acta} 
manualmente, cubrindo as estadísticas do encontro.
     \item O \textbf{acta} asínase polo árbitro e un representante de cada equipo.
     \item Cada clube guarda unha copia do \textbf{acta} e o árbitro transalada a súa ata 
a federación.
     \item A federación revisa o \textbf{acta}, comprobando que os xogadores que xogaron 
non estaban sancionados, se pertence ao equipo e copiando tódolos datos recollidos, na 
aplicación de xestión ou a fólla de cálculo da que dispoñan.
    \end{enumerate}
  
  É por isto polo que se decidíu crear unha aplicación móbil que permita que os árbitros 
xestionen as súas actas de forma electrónica, directamente dende o seu teléfono móbil, 
nunha aplicación multidispositivo baseada en tecnoloxías web, permitindo incluso realizar 
ditas actas sen conexión a internet, algo que hoxe en día ningunha aplicación ofrece no 
mercado nacional.

  Esta aplicación móbil integrarase tamén nun sistema de xestión de 
competicións os árbitros poidan cubrir as actas e publicar as estadísticas e 
resultados directamente na web da federación, a través do seu sistema de 
xestión.

    \section{VACmatch}
    \santiagosays{Esto mola, e mola contalo, pero eu creo que noutra orde:
      \begin{enumerate}
      \item Problema: as cousas estaban mal
      \item Antecedentes/outras cousas que existen: excel, e outras
        alternativas que non solucionan todo o problema e das que se
        fala máis no seguinte capítulo
      \item Solución: había que facer algo para arreglalo
        \begin{enumerate}
        \item VACmatch Mobile: empezouse por aquí porque era onde máis
          había que facer
        \item VACmatch: montouse unha iniciativa empresarial para
          levar isto ao mercado e seguir desenvolvendo cousas
        \end{enumerate}
      \end{enumerate}}

    VACmatch é unha iniciativa empresarial xurdida na Universidade da Coruña para 
mellorar a xestión de competicións deportivas a través dunha serie de 
aplicacións entre as que se atopa este proxecto.

    A iniciativa recibíu o pasado ano a calificación de \emph{Iniciativa Empresarial de 
Base Tecnolóxica (IEBT)} recoñecendo o seu grao de innovación así como participou en 
diversos programas de apoio a ideas emprendedoras como \emph{Yuzz} e \emph{Telefónica 
Galicia OpenFuture\_}.

   Actualmente a iniciativa atópase afincada no \emph{Viveiro de empresas da Universidade 
da Coruña}

    \section{Resumo do proxecto}
    O proxecto desenvolto componse de dúas partes diferenciadas que permite a 
xestión das actas dos encontros por parte das federacións deportivas.
    
    \subsection{VACmatch Mobile}
    É unha aplicación móbil híbrida realizada con tecnoloxías web co fin de 
poder utilizala en calquera plataforma, tanto a través da web como nun móbil 
Android ou IOS e que permitirá aos árbitros das competiciones realizar 
todas as xestións coas actas dos encontros dende o seu teléfono.

  \begin{description}
    \item [Lista de actas]
    Esta aplicación permite aos árbitros dos encontros no seu teléfono 
das actas dos partidos que teñen que dirixir, coa localización e a data dos 
mesmos e que se actualizan de forma automática cando se reasignan ou se 
cancelan.

    \item [Convocatoria de xogadores]
    Unha vez o árbitro chega ao encontro pode seleccionar na aplicación os 
xogadores que asistiron ao mesmo únicamente con un click, introducir a algún 
novo se o desexa ou editar datos como o dorsal dun xogador.

    \item [Xestión de actas]
    Unha vez comezado o encontro a aplicación permitirá introducir os diversos 
eventos que ocorren no mesmo como infraccións, goles ou tarxetas de forma que é 
moi sinxelo engadir novos deportes e eventos.
    
    \item [Sinatura de actas]
    Para rematar o encontro, o árbitro poderá engadir comentarios a mesma acta 
e tanto él como un xogador de cada equipo poderán asinar a acta con un código 
PIN.

    \item [Actas offline]
    O árbitro poderá crear actas incluso aínda que non tivese sincronizados 
todos os datos do partido, permitindo cubrir as actas incluso no peor escenario 
posible.

  \end{description} 

    \subsection{VACmatch}
    A aplicación móbil explicada antes tamén se atopa integrada nunha 
aplicación web de xestión de competicións a través da cal, as federacións poden 
xestionar as súas competiciones, modificar o calendario, engadir novos 
xogadores ou equipos, xestionar arbitraxes, etc.

    A parte que corresponde a xestión de actas de encontros tamén pertence a 
este proxecto e permite que a federación cree as actas, os árbitros a 
sincronicen nos seus teléfonos e unha vez cubertas, todos os datos sexan 
publicados automáticamente nesta aplicación de xestión.

    Esto permite manter os resultados e as clasificacións actualizadas en todo 
momento nas federacións se apenas intervención humana, aforrando un enorme 
traballo na revisión das actas e no transporte das mesmas ata a sede da 
federación.


  \subsection{Estrutura da memoria}
  
  \todo{Engadir estrutura}
  \santiagosays{Acórdate, pero primeiro ten sentido ter a estrutura clara antes de poñerse con isto :-)}
  
  
%%% Local Variables:
%%% mode: latex
%%% TeX-master: "../root"
%%% End:
