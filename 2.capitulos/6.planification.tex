\chapter{Planificación e seguimento}
\minitoc
% \label{chap:Planificacioneseguimento}
% \vspace{0.5cm}

%%%%%%%%%%%%%%%%%%%%%%%%%%%%%%%%%%%%%%%%%%%%%%%%%%%%%%%%%%%%%%%%%%%%%%%%%%%%%%%%
% Objetivo:                        %
%%%%%%%%%%%%%%%%%%%%%%%%%%%%%%%%%%%%%%%%%%%%%%%%%%%%%%%%%%%%%%%%%%%%%%%%%%%%%%%%

  \lettrine{E}{n} este capítulo...

  \section{VACmatch validación de negocio. Agosto 2015 - Outubro 2015}
  blabla empresa, clientes, GOF y Yuzz, torneo de testing

    \subsection{MVP}
    Validación de usuario con tests de laboratorio sobre versión sin funcionalidade
    1 mes

    \subsection{MVP funcional}
    Creación de APP sinxela para tests de campo reais.
    2 mes
      \paragraph{Planificación temporal}
      \paragraph{Definición da iteración}
      \paragraph{Feedback}
      \paragraph{Tarefas e seguimento}

  \section{VACmatch desenvolvemento de produto. Outubro 2015 - Xaneiro 2016}
  Ronda de inversión, incrición cusl, blog vacmach

    \subsection{1ª iteración. Creación do proxecto}
    Aprender PouchDB, deseño da estructura e arquitectura, deseño modelo de datos, listar 
  actas.

    \subsection{2ª iteración. Xestión de actas}
    Ver o resumo dun acta xenérica, cronómetro e control do tempo 

    \subsection{3ª iteración. Eventos}
    Xestión de eventos xenérica, sport events e control events, actualizar resultados na 
  acta, listar e borrar eventos, convocar xogadores,, listar eventos

    \subsection{4ª iteración. Xestión de usuarios e creación offline de actas}
    login, logout, metese staff, crear actas offline
    Memoria, 3 primeros apartados

    \subsection{5ª iteración. Sinaturase}
    Sinaturas con PIN, engadir incidencias
    do sprint anterior: crear usuario, crear árbitro ao crear user

  \section{VACmatch de empresa a comunidade. Xaneiro 2016 - Maio 2016}
    \subsection{6ª e 7ª iteración. Optimización e melloras}
    Refactor de servizos creando un xenérico, crear clases para cada entidade e simplificar o código
    creación aleatoria de ids

    \subsection{8ª iteración. Testing e integración continua}
    Tests servicios, travis e confirmar contrasinal e PIN

    \subsection{9ª e 10ª iteración. Inxección de dependencias}
    Corrección de erros na CI, engadir textos de error, creación de snackbar para erros e comunicacións
    Inxección de dependencias -> motivo dependencias circulares
    Estados no report: pasar de isFinished -> Ready, Started e Finished

    \subsection{Release 0.2.0: Usabilidade en menús}
    Links en menus e engadir información e documentación en github (instalación, DB, etc)

    \subsection{Release 0.2.1: I18n e app híbrida}
    React intl, cordova, documentación instalación en Android

    \subsection{Release 0.2.2: Imáxe corporativa e revisión de erros}
    Imaxe corporativa VACmatch, bugfix
    Memoria a saco

    \subsection{Release 0.3.0: Usabilidade móbil e entrega continua}
    Por cordova: dialogos -> ventás, transición carga, migrar taiga.io, Travis + Docker