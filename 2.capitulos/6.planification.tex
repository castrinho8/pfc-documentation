\chapter{Planificación e seguimento}
\minitoc
\label{chap:Planificacioneseguimento}
\vspace{0.5cm}

%%%%%%%%%%%%%%%%%%%%%%%%%%%%%%%%%%%%%%%%%%%%%%%%%%%%%%%%%%%%%%%%%%%%%%%%%%%%%%%%
% Objetivo:                        %
%%%%%%%%%%%%%%%%%%%%%%%%%%%%%%%%%%%%%%%%%%%%%%%%%%%%%%%%%%%%%%%%%%%%%%%%%%%%%%%%

  \lettrine{E}{n} este capítulo...

  \section{MVP}
  Validación de usuario con tests de laboratorio sobre versión sin funcionalidade
  1 mes

  \section{MVP funcional}
  Creación de APP sinxela para tests de campo reais.
  1 mes
    \paragraph{Planificación temporal}
    \paragraph{Definición da iteración}
    \paragraph{Feedback}
    \paragraph{Tarefas e seguimento}
  
  \section{1ª iteración. Creación do proxecto}
  Aprender PouchDB, deseño da estructura e arquitectura, deseño modelo de datos, listar 
actas.
  
  \section{2ª iteración. Xestión de actas}
  Ver o resumo dun acta xenérica, cronómetro e control do tempo 
  
  \section{3ª iteración. Eventos}
  Xestión de eventos xenérica, sport events e control events, actualizar resultados na 
acta, listar e borrar eventos
  Comezo da memoria
  
  \section{4ª iteración. Xestión de usuarios e creación e sinatura de actas}
  Engádense primeros test de servicios
  Memoria, 3 primeros apartados
  
  \section{5ª iteración. Internacionalización e usabilidade}
  Finalización da memoria, revisión dos menús, i18n e corrección de erros.

  \section{6ª iteración. Backend remoto e corrección de erros}
  
