\chapter{Planificación e seguimento}
\minitoc
% \label{chap:Planificacioneseguimento}
% \vspace{0.5cm}

%%%%%%%%%%%%%%%%%%%%%%%%%%%%%%%%%%%%%%%%%%%%%%%%%%%%%%%%%%%%%%%%%%%%%%%%%%%%%%%%
% Objetivo:                        %
%%%%%%%%%%%%%%%%%%%%%%%%%%%%%%%%%%%%%%%%%%%%%%%%%%%%%%%%%%%%%%%%%%%%%%%%%%%%%%%%

  \lettrine{N}{e} ste capítulo detallaremos a planificación e o seguimento do 
proxecto, un proxecto que por diversas circunstancias se dividiu principalmente 
en tres grandes etapas, as dúas primeiras mentres VACmatch era unha iniciativa 
emprendedora e a última durante a cal se comezou unha conversión da iniciativa 
cara un proxecto comunitario de software libre.

\santiagosays{Ao principio tamén quería ser un proxecto de SwL, non?
  Igual isto hai que explicalo mellor nalgures. Porque antes deste
  capítulo todo é falar de VACmatch como unha iniciativa emprendedora;
  e, bueno, agora mesmo non estamos emprendendo moito nin moi lean.}

  \begin{description}
    \item [Agosto 2015 - Outubro 2015] VACmatch. Validación de negocio.
    \item [Outubro 2015 - Xaneiro 2016] VACmatch. Desenvolvemento de produto.
    \item [Xaneiro 2016 - Xuño 2016] De empresa a comunidade.
  \end{description}


  \section{VACmatch validación de negocio. Xullo 2015 -- Novembro 2015}
  \santiagosays{Porqué lle chamas VACmatch <nome> á sección? (Pregunto, simplemente)\\
    Igual queda mellor estilísticamente poñer as datas entre
    parénteses)}

  A duración de esta etapa é de aproximadamente 4 meses e ven determinada polos 
primeiros pasos de VACmatch como iniciativa empresarial e que levan a orientar 
o desenvolvemento do produto cara o cliente, comezando con unha serie de 
prototipos para coñecer as suas necesidades e validar a idea de 
negocio.

  Durante este periodo tamén se planificou a organización dun torneo de fútbol 
sala o finais do mes de Outubro coa idea de probar nun entorno 
controlado os primeiros prototipos desenvoltos.

  Durante o primeiro mes planificase a realización de un prototipo visual co 
fin de comprobar a usabilidade e consolidar os requisitos dos clientes.
  De seguido, plantéxase crear un pequeno prototipo funcional, un Mínimo 
Producto Viable (MVP) na metodoloxía Lean Startup, co obxectivo de testear as 
necesidades dos clientes e definir o produto final a desenvolver.

    \subsection{Prototipo visual}

      \subsubsection{Planificación temporal}
      Esta iteración dura un total de 4 semanas de desenvolvemento entre o 15 
de Xullo e o 16 de Agosto e realizase unha visita semanal ao cliente para obter 
feedback e mostrarlle a evolución do prototipo.

      \subsubsection{Definición da iteración}
      Durante este periodo de un mes de duración planificouse o desenvolvemento 
de unha aplicación moi sinxela e sen funcionalidade, que únicamente permitise 
analizar a usabilidade do sistema e comprobar si é factible adaptar o proceso 
de creación de un acta deportiva, a unha aplicación móbil.

    Así mesmo, ao longo do período realizaranse ata tres visitas a federación 
coa que se traballou dende o primeiro momento para comprobar a experiencia de 
un futuro usuario real da aplicación e obter feedback para futuras melloras.

      \subsubsection{Reporte e feedback}
      Durante as visitas as federacións obtivéronse diversas críticas 
      \begin{itemize}
        \item Facer interactiva a aplicación e non mostrar grandes táboas con 
datos.
        \item Todas as accións deben xirar ao redor da acta.
        \item Crear partidos cando non hai cobertura.
      \end{itemize}

      \subsubsection{Tarefas e seguimento}

      As tarefas de esta iteración son as seguinte:

      \begin{itemize}
        \item Crear esqueleto da aplicación.
        \item Como árbitro quero poder consultar as próximas actas a 
cubrir.
        \item Como árbitro quero poder consultar as actas xa cubertas.
        \item Como árbitro quero poder editar un acta.
        \item Como árbitro quero poder ver un acta.
        \item Como árbitro quero poder ver os xogadores de ambos equipos.
        \item Como árbitro quero poder engadir un evento.
        \item Como árbitro quero poder rematar ou suspender un partido.
        \item Como árbitro quero poder logearme.
        \item Como árbitro quero poder seleccionar cales xogadores de cada 
equipo se atopan no encontro.
        \item Como árbitro quero poder borrar un evento.
        \item Estudio sobre React e Flux.
       \end{itemize}

      Planificaronse X horas, fixéronse X por isto

    \subsection{MVP funcional}
    Unha vez finalizadas as probas visuales e de usabilidade procedese 
a planificar o desenvolvemento para adaptar o prototipo e engadirlle 
funcionalidade sinxela, sen validacións nin controles co obxectivo de obter un 
prototipo funcional que poida ser utilizado por usuarios reais nun entorno 
controlado.

      \subsubsection{Planificación temporal}
      Esta iteración dura un total de 3 meses e desenvólvese entre o 16 de 
Agosto e o 15 de Novembro.

      \subsubsection{Definición da iteración}
      \subsubsection{Feedback}
      \begin{itemize}
      \item Poder ver os eventos de forma sinxela dende a vista de fin
        de partido.
      \end{itemize}
      \subsubsection{Tarefas e seguimento}


  \section{VACmatch desenvolvemento de produto. Novembro 2015 - Xaneiro 2016}
  Ronda de inversión\santiagosays[]{Oh, e isto cando foi? Eu non recordo que houbera inversores naquel sitio}, incrición CUSL\santiagosays[]{Siglas.., expándeas ou pon nota ao pé coa referencia á web; ou incluso ambas cousas}, blog VACmatch\santiagosays[]{Non queres facer spam da web?}

  \santiagosays{A partir de aquí non leín, que parece que está aínda moi work in progress}
  \ProTip{Fíxate en que tes algúns caracteres non se mostran correctamente. Se queres poñer unha flecha, en latex tes o comando \texttt{\textbackslash rightarrow} (en modo matemático) (xenera: $\rightarrow$)}

    \subsection{1ª iteración. Creación do proxecto}
    Aprender PouchDB, deseño da estructura e arquitectura, deseño modelo de datos, listar 
  actas.
      \subsubsection{Planificación temporal}
      \subsubsection{Definición da iteración}
      \subsubsection{Feedback}
      \subsubsection{Tarefas e seguimento}

        \paragraph{Análise final de requisitos}
        \paragraph{Definir modelo de datos}
        \paragraph{Estudo das tecnoloxías}
          \subparagraph{React e Flux}
          \subparagraph{PouchDB}
          \subparagraph{Redmine}
        \paragraph{Mockups da aplicación}
        \paragraph{Mockups da aplicación}
          \subparagraph{}
          \subparagraph{}
          \subparagraph{}

    \subsection{2ª iteración. Xestión de actas}
    Ver o resumo dun acta xenérica, cronómetro e control do tempo 
      \subsubsection{Planificación temporal}
      \subsubsection{Definición da iteración}
      \subsubsection{Feedback}
      \subsubsection{Tarefas e seguimento}

    \subsection{3ª iteración. Eventos}
    Xestión de eventos xenérica, sport events e control events, actualizar resultados na 
  acta, listar e borrar eventos, convocar xogadores,, listar eventos
      \subsubsection{Planificación temporal}
      \subsubsection{Definición da iteración}
      \subsubsection{Feedback}
      \subsubsection{Tarefas e seguimento}

    \subsection{4ª iteración. Xestión de usuarios e creación offline de actas}
    login, logout, metese staff, crear actas offline
    Memoria, 3 primeros apartados
      \subsubsection{Planificación temporal}
      \subsubsection{Definición da iteración}
      \subsubsection{Feedback}
      \subsubsection{Tarefas e seguimento}

    \subsection{5ª iteración. Sinaturase}
    Sinaturas con PIN, engadir incidencias
    do sprint anterior: crear usuario, crear árbitro ao crear user
      \subsubsection{Planificación temporal}
      \subsubsection{Definición da iteración}
      \subsubsection{Feedback}
      \subsubsection{Tarefas e seguimento}


  \section{VACmatch de empresa a comunidade. Xaneiro 2016 - Maio 2016}
    \subsection{6ª e 7ª iteración. Optimización e melloras}
    Refactor de servizos creando un xenérico, crear clases para cada entidade e simplificar o código
    creación aleatoria de ids
      \subsubsection{Planificación temporal}
      \subsubsection{Definición da iteración}
      \subsubsection{Feedback}
      \subsubsection{Tarefas e seguimento}

    \subsection{8ª iteración. Testing e integración continua}
    Tests servicios, travis e confirmar contrasinal e PIN
      \subsubsection{Planificación temporal}
      \subsubsection{Definición da iteración}
      \subsubsection{Feedback}
      \subsubsection{Tarefas e seguimento}

    \subsection{9ª e 10ª iteración. Inxección de dependencias}
    Corrección de erros na CI, engadir textos de error, creación de snackbar para erros e comunicacións
    Inxección de dependencias -> motivo dependencias circulares
    Estados no report: pasar de isFinished -> Ready, Started e Finished
      \subsubsection{Planificación temporal}
      \subsubsection{Definición da iteración}
      \subsubsection{Feedback}
      \subsubsection{Tarefas e seguimento}

    \subsection{Release 0.2.0: Usabilidade en menús}
    Links en menus e engadir información e documentación en github (instalación, DB, etc)
      \subsubsection{Planificación temporal}
      \subsubsection{Definición da iteración}
      \subsubsection{Feedback}
      \subsubsection{Tarefas e seguimento}

    \subsection{Release 0.2.1: I18n e app híbrida}
    React intl, cordova, documentación instalación en Android
      \subsubsection{Planificación temporal}
      \subsubsection{Definición da iteración}
      \subsubsection{Feedback}
      \subsubsection{Tarefas e seguimento}

    \subsection{Release 0.2.2: Imáxe corporativa e revisión de erros}
    Imaxe corporativa VACmatch, bugfix
    Memoria a saco
      \subsubsection{Planificación temporal}
      \subsubsection{Definición da iteración}
      \subsubsection{Feedback}
      \subsubsection{Tarefas e seguimento}

    \subsection{Release 0.3.0: Usabilidade móbil e entrega continua}
    Por cordova: dialogos -> ventás, transición carga, migrar taiga.io, Travis + 
Docker
      \subsubsection{Planificación temporal}
      \subsubsection{Definición da iteración}
      \subsubsection{Feedback}
      \subsubsection{Tarefas e seguimento}


%%% Local Variables:
%%% mode: latex
%%% TeX-master: "../root"
%%% End:
