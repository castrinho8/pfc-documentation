\chapter{Metodoloxía}
\minitoc
% \label{chap:Metodoloxia}
% \vspace{0.5cm}

%%%%%%%%%%%%%%%%%%%%%%%%%%%%%%%%%%%%%%%%%%%%%%%%%%%%%%%%%%%%%%%%%%%%%%%%%%%%%%%%
% Objetivo:                        %
%%%%%%%%%%%%%%%%%%%%%%%%%%%%%%%%%%%%%%%%%%%%%%%%%%%%%%%%%%%%%%%%%%%%%%%%%%%%%%%%

  \lettrine{N}{este} capítulo imos analizar as diversas metodoloxías utilizadas para a 
xestión do proxecto e explicar a adaptación das mesmas que finalmente se 
utilizou.

  Comezaremos falando da metodoloxía orientada ao modelo de negocio xa que, 
como se indica na introdución, este proxecto xurdíu dentro de unha iniciativa 
empresarial e polo tanto dende o primeiro momento traballouse orientado cara o 
cliente, facendo que sexa nun dos piares do desenvolvemento.

  Seguidamente comentaremos as metodoloxías áxiles que serviron de base para 
definir a metodoloxía utilizada finalmente, unha adaptación das mencionadas e 
que tamén se indica para rematar o capítulo.

  \section{Lean Startup}
  Lean Startup\cite{book:leanstartup} é unha metodoloxía para abordar o 
lanzamento de negocios e produtos a través da validación, a experimentación e a 
iteración no lanzamento dos mesmos co fin de acortar o ciclo de desenvolvemento.

  É unha metodoloxía de traballo moi habitual nas startups que se centra na 
idea de \emph{Crear - Medir - Aprender}, desenvolvendo pequenos produtos e 
realizando tests de mercado reais con verdadeiros clientes co fin de medir o 
seu grao de satisfación e aprender para mellorar o produto en seguintes 
iteracións.

  Habitualmente céntrase na idea de crear un MVP (Minimum Viable Product), 
unha versión do producto que permite os desenvolvedores recoller co mínimo 
esforzo a máxima cantidade de coñecemento validado por parte dos clientes, 
evaluando as hipóteses de se os clientes realmente estarían dispostos a pagar 
polo producto e implicando a dito cliente no desenvolvemento do produto.

  \section{eXtreme Programming}
  \label{sec:method:extreme}
  eXtreme Programming\cite{book:agile} é unha metodoloxia de desenvolvemento 
áxil e incremental baseada na integración do cliente no desenvolvemento así como 
na simplicidade do código.

  A metodoloxia aposta por facer as cousas sinxelas, sen preocuparse por ter 
que facer un pequeno traballo por adaptalas se é preciso, fronte a idea 
tradicional de facer un gran traballo para quizáis nunca chegar a utilizar 
parte do mesmo.

  As entregas funcionais son frecuentes e outras características como a 
importancia de introducir a programación en parellas para reducir o número de 
erros que se producen ao programar.

  Por último aboga por introducir o TDD (Test Driven 
Development)\cite{book:cleancode}, implementando primeiro os tests, 
verificando que fallan para a continuación implementar o código que fai que 
pasen correctamente os mesmos.
  A idea é que os requisitos sexan convertidos a probas e deste modo cando os 
tests se pasen, poderemos garantizar que o código cumple os requisitos.

  \section{Scrum}
  Scrum\cite{book:scrum} tamén é unha metodoloxía incremental de desenvolvemento 
cunha serie de roles definidos para o proceso, cada un coas súas 
responsabilidades e que divide o proxecto en varios \emph{Sprints} que son 
ciclos de desenvolvemento.

  Cada un deles ten unha duración definida polo equipo de, habitualmente, 
entre unha e catro semanas, proporcionando un incremento de software entregable 
ao final de cada \emph{Sprint}.

  A totalidade das tarefas do proxecto atópanse definidas e priorizadas nunha 
lista chamada \emph{Product Backlog}. Para cada sprint, selecciónanse aquelas 
tarefas que determinarán a lista a implementar durante a presente iteración, o
\emph{Sprint Backlog}, e que non pode variar ata rematar o sprint.

  Durante todo o ciclo de traballo realízanse reunións diarias para comprobar o 
estado do proxecto así como outras ao finalizar e ao comezar os sprints, co 
fin de analizar a iteración anterior e planificar a seguinte, facendo un 
seguimento continuo do proxecto e facilitando a adaptación do mesmo a posibles 
novos requisitos.

  \section{Adaptación da metodoloxía}

    \subsection{Desenvolvemento orientado ao cliente}
      O proxecto ten lugar dentro dunha iniciativa empresarial polo que se 
decidiu utilizar un modelo de desenvolvemento orientado ao cliente en todo 
momento, baseandose no pilar central da metodoloxia \emph{Lean Startup}.

    Para isto realizáronse diversas visitas as federacións para comprobar as 
súas necesidades a través dunha serie de entrevistas estructuradas para 
coñecer os problemas e a súa prioridade á hora de resolvelos.

    Do mesmo modo realizáronse dous prototipos, un primeiro únicamente 
con plantillas HTML\footnote{\emph{HyperText Markup Language}.} para testear a 
organización da interfaz de usuario e un 
segundo xa funcional para comprobar a resposta dos usuarios finais ante o seu 
funcionamento.

    \subsection{Sprints con backlog adaptable}
    A organización do desenvolvemento organizouse de xeito moi similar á idea 
proposta en \emph{Scrum}, dividindo o proceso en sprints, pequenas iteracións 
de dúas ou tres semanas de duración e que cada unha proporciona unha serie de 
novas funcionalidades.

    Cada sprint comeza cunha reunión de aproximadamente 30/45 minutos de 
duración na que realizar a planificación do mesmo en función do 
traballo realizado no sprint anterior, o que permite realizar melloras nas 
previsións segundo o aprendido dos anteriores.

    A diferencia do proposto por Scrum, decidíuse optar por sprints de duración 
variable e cun backlog adaptable ao longo do sprint según as necesidades xa que 
proporciona unha maior flexibilidade e liberdade.

    \subsection{Reunións semanais}
    Todas as semanas faise unha reunión de 30/45 minutos de duración na que 
analizar o realizado na semana anterior e comprobar o seguimento da iteración 
co fin de atopar desviacións e corrixilas.

    Cando unha reunión semanal coincide co fin de un sprint, dita reunión 
serve para realizar a planificación do seguinte sprint de xeito moi similiar ás 
reunións de sprint que se realizan en \emph{Scrum}.

    \subsection{Reunións diarias}
    Ao comezar o día realízase unha análise duns 10 minutos de duración para 
revisalo realizado no día anterior e planificar de forma máis concreta o que 
se vai facer ese mesmo día.

    \subsection{Releases}
    Durante o desenvolvemento do proxecto trátase de aplicar a idea de realizar 
unha serie de pequenos entregables en cada iteración.

    Todas as entregas ao finalizar unha iteración son totalmente funcionais 
pero non todas son versións entregables reais para ser postas en produción.

    Durante o desenvolvemento producíronse 4 entregas (\emph{releases}) 
totalmente funcionais, a primeira foi un prototipo, a segunda foi a versión 
real do proxecto, a terceira incorporou os tests e a cuarta correxiu diversas 
características para asegurar unha primeira versión estable.

    \subsection{Simplicidade}
    Utilizouse o principio de simplicidade que promove \emph{eXtreme 
Programming} durante todo o desenvolvemento baixo a máxima de implementar 
únicamente o imprescindible en cada momento, sempre pensando en programar 
para hoxe e non para mañá.

    A idea fundaméntase en realizar refactorizacións de código para engadir 
novas funcionalidades a medida que son necesarias en lugar de invertir 
demasiado tempo na planificación e implementación de funcións que se supoñen 
necesarias e, algunhas das cales, é probable que non sexan utilizadas
finalmente.

    \subsection{Tests}
    A importancia de creación de tests automatizados está totalmente 
demostrada, atopándose en auxe metodoloxías como
TDD~(Test~Driven~Development)%
\footnote{Desenvolvemento dirixido polos tests.} %
ou BDD~(Behaviour~Driven~Development)%
\footnote{Desenvolvemento dirixido polo comportamento} %
que tratan de dirixir o desenvolvemento a través dos tests e
que son realizados antes da implementación da funcionalidade.

    Durante a primeira parte do desenvolvemento non se aplicou ningunha destas 
metodoloxías pero a partir da terceira \emph{release} e da integración 
dos primeiros tests, decidíuse optar por aplicar TDD no desenvolvemento, 
realizando probas unitarias nos servicios utilizados.

    \subsection{Fluxo de contribución ao proxecto}
    O fluxo de traballo utilizado dende o primeiro día trata de simular o 
traballo diario de equipo e permite controlar a evolución do código de xeito 
máis ordenado.

    Disponse de unha rama \emph{master} na que se atopa a versión estable de 
desenvolvemento así como de unha rama \emph{development} que é máis inestable e 
que ao final de cada sprint, é integrada dentro de \emph{master}.

    Unha nova funcionalidade ou erro é resolto nunha nova rama independente,
creada a partir de \emph{development} e tratando que todas estas novas 
funcionalidades sexan independentes entre si.

    Así mesmo tratase de que todos os \emph{commits} sexan funcionais e o máis 
independentes posibles, evitando ter algún que non compile ou que non pase 
os tests.

    Posteriormente realizase unha \emph{pull request}\footnote{Unha \emph{pull
request} é unha petición para integrar unha rama de Git en outra a través de
GitHub, un mecanismo habitual para engadir novas funcionalidades ou correxir
erros.} a través do mecanismo que proporciona o repositorio
de código de GitHub, esperando que alguén revise o código para ser
integrado na rama de desenvolvemento.

    Cada certo tempo revísanse as \emph{pull requests} abertas, analízase o
codigo e se todo é correcto, acéptase.

    Ao final de cada sprint realízase una nova \emph{pull request} para
integrar a funcionalidade creada no sprint actual e que se atopa na rama de
desenvolvemento, dentro da rama estable.

    Dende a introducción de tests no proxecto, todo código subido ao 
repositorio é analizado a través dun sistema de integración continua
---que se comenta con máis detalle na Sección~\ref{sec:travis},--- e que comproba se os cambios 
engadidos pasan os tests ou non, e avisan por correo 
electrónico do resultado.


%%% Local Variables:
%%% mode: latex
%%% TeX-master: "../root"
%%% End:
