\chapter{Metodoloxía}
\minitoc
\label{chap:Metodoloxia}
\vspace{0.5cm}

%%%%%%%%%%%%%%%%%%%%%%%%%%%%%%%%%%%%%%%%%%%%%%%%%%%%%%%%%%%%%%%%%%%%%%%%%%%%%%%%
% Objetivo:                        %
%%%%%%%%%%%%%%%%%%%%%%%%%%%%%%%%%%%%%%%%%%%%%%%%%%%%%%%%%%%%%%%%%%%%%%%%%%%%%%%%

  \lettrine{N}{este} capítulo imos analizar as diversas metodoloxías utilizadas para a 
xestión do proxecto e explicar a adaptación das mesmas que finalmente se utilizóu.

  \section{Lean Startup}

  \section{eXtreme Programming}

  \section{Scrum}

  \section{Adaptación da metodoloxía}
    \subsection{Desenvolvemento orientado ao cliente}
      Testeos con clientes reais, analisis das súas necesidades, priorización por parte 
do cliente, MVP.

    \subsection{Sprints}
    Cada dúas semanas con entregas funcionáis.

    \subsection{Reunións semanáis}
    Análise da semana anterior e comprobación do seguimento da iteración.
    Se rematóu un sprint, esta reunión serve de planificación do seguinte.

    \subsection{Reunións diaria}
    Análise 10 minutos do día anterior y do que se vai a facer no presente.

    \subsection{Releases}
    2 entregas 100 \% funcionais.

    \subsection{Simplicidade}
    Para manter a simplicidade é precisa a refactorización de código. Facer únicamente o 
imprescindible en cada momento.
    Programar para hoxe e non para mañá

    \subsection{Tests}
    Probas unitarias (neste caso solo de servicios).

    \subsection{Fluxo de traballo}
    PR en github con code review e integración continua

