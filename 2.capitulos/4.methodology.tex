\chapter{Metodoloxía}
\minitoc
\label{chap:Metodoloxia}
\vspace{0.5cm}

%%%%%%%%%%%%%%%%%%%%%%%%%%%%%%%%%%%%%%%%%%%%%%%%%%%%%%%%%%%%%%%%%%%%%%%%%%%%%%%%
% Objetivo:                        %
%%%%%%%%%%%%%%%%%%%%%%%%%%%%%%%%%%%%%%%%%%%%%%%%%%%%%%%%%%%%%%%%%%%%%%%%%%%%%%%%

  \lettrine{N}{este} capítulo imos analizar as diversas metodoloxías utilizadas para a 
xestión do proxecto e explicar a adaptación das mesmas que finalmente se utilizóu.

  \section{Lean Startup}
  Lean Startup é unha metodoloxía para abordar o lanzamento de negocios e 
produtos a través da validación, a experimentación e a iteración no lanzamento 
dos mesmos para acortar o ciclo de desenvolvemento.

  É unha metodoloxía de traballo moi habitual nas startups que se centra na 
idea de \emph{Crear - Medir - Aprender}, desenvolvendo pequenos produtos e 
realizando tests de mercado reais con verdadeiros clientes co fin de medir o 
seu grao de satisfación e aprender para mellorar o produto en seguintes 
iteracións.
  
  Habitualmente céntrase na idea de crear un MPV (Mínimo Producto Viable), unha 
versión do producto que permite os desenvolvedores recoller co mínimo esforzo a 
máxima cantidade de coñecemento validado por parte dos clientes, evaluando as 
hipóteses de se os clientes realmente estarían dispostos a pagar polo 
producto e implicando a dito cliente no desenvolvemento de dito produto.
  
  \section{eXtreme Programming}
  eXtreme Programming é unha metodoloxia de desenvolvemento áxil e incremental 
baseada na integración do cliente no desenvolvemento así como na simplicidade 
do código, apostando por facer cousas sinxelas e ter que facer un pequeno 
traballo por modificalo se é preciso, fronte a facer un gran traballo para 
quizáis nunca utilizalo.

  As entregas funcionáis son frecuentes e outras características como a 
importancia de introducir a programación en parellas para reducir o número de 
erros que se producen ao programar.

  Por último aboga por introducir o TDD (Test Driven Development), 
implementando primeiro os tests, verificar que fallan e a continuación 
implementar o código que fai que pasen os tests.
  A idea é que os requisitos sexan convertidos a probas e de este modo cando os 
tests se pasen, poderemos garantizar que o código cumple os requisitos.

  \section{Scrum}
  Scrum tamén é unha metodoloxía incremental de desenvolvemento cunha serie de 
roles definidos para o proceso, cada un coas súas responsabilidades e que divide 
o proxecto en varios \emph{Sprints} que son ciclos de desenvolvemento.

  Cada un de eles ten unha duración de entre unha e catro semanas e que é 
definida polo equipo e cada unha das cales proporciona un incremento de 
software entregable.

  A totalidade das tarefas do proxecto atópanse definidas e priorizadas no 
\emph{Product Backlog} e para cada sprint selecciónanse aquelas que determinarán 
o \emph{Sprint Backlog}, que serán implementadas durante dito sprint e que non 
poden variar ata rematar.

  Durante todo o ciclo de traballo realízanse reunións diarias para comprobar o 
estado do proxecto así como outras ao fin e ao comezo dos sprints co fin de 
analizar o anterior e planificar o seguinte.

  \section{Adaptación da metodoloxía}

    \subsection{Desenvolvemento orientado ao cliente}
      O proxecto ten lugar dentro dunha iniciativa empresarial polo que se 
decidíu utilizar un modelo de desenvolvemento orientado ao cliente en todo 
momento, baseandose no pilar central da metodoloxia \emph{Lean Startup}.

    Para isto realizáronse diversas visitas as federacións para comprobar as 
súas necesidades a través dunha serie de entrevistas estructuradas para 
coñecer os problemas e a súa prioridade a hora de resolvelos.

    Do mesmo modo realizáronse dous prototipos, un primeiro únicamente 
con plantillas HTML para testear a organización da interfaz de usuario e un 
segundo xa funcional para comprobar a resposta dos usuarios reais ante o seu 
funcionamento.

    \subsection{Sprints con backlog adaptable}
    A organización do desenvolvemento organizouse de xeito moi similar a idea 
proposta en \emph{Scrum}, dividindo o proceso en sprints, pequenas iteracións 
de dúas semanas de duración e que cada unha proporciona unha serie de novas 
funcións totalmente .

    Cada sprint comeza con unha reunión de aproximadamente 30/45 minutos de 
duración na que realizar a planificación do mesmo en función do 
traballo realizado no sprint anterior o que permite realizar melloras nas 
previsións según o aprendido nos anteriores.

  \todo{falar do backlog que se adapta durante o sprint, non como en scrum que 
é fixo}
    \subsection{Reunións semanáis}
    Todas as semanas faise unha reunión de 30/45 minutos de duración na que 
analizar o realizado na semana anterior e comprobar o seguimento da iteración 
co fin de atopar desviacións e correxilas.

    Cando unha reunión semanal coincide co fin de un sprint, dita reunión 
sirve para realizar a planificación do seguinte sprint de xeito moi similiar as 
reunións de sprint que se realizan en \emph{Scrum}.

    \subsection{Reunións diarias}
    Ao comezar o día realízase unha análise de uns 10 minutos de duración para 
revisar o realizado no día anterior e planificar de forma máis concreta o que 
se vai facer ese mesmo día.

    \subsection{Releases}
    Durante o desenvolvemento do proxecto trátase de aplicar a idea de realizar 
unha serie de pequenos entregables en cada iteración.
    
    Todas as entregas ao finalizar unha iteración son totalmente funcionais 
pero non todas son versións entregables reais para ser postos en producción.

    Durante o desenvolvemento producíronse 3 entregas (\emph{releases}) 
totalmente funcionáis, a primeira foi un prototipo, a segunda foi a versión 
real do proxecto e a terceira incorporóu tests e diversas características para 
asegurar unha primeira versión estable.

    \subsection{Simplicidade}
    Utilizouse o principio de simplicidade que promove \emph{eXtreme 
Programming} durante todo o desenvolvemento baixo a máxima de implementar 
únicamente o imprescindible en cada momento, sempre pensando en programar para 
hoxe e non para mañá.

    A idea fundamentase en realizar refactorizacións de código para engadir 
novas funcionalidades a medida que son necesarias en lugar de invertir 
demasiado tempo na planificación e implementación de funcións que se supoñen 
necesarias e, algunhas das cales, é probable que non sexan necesarias 
finalmente.

    \subsection{Tests}
    A importancia de creación de tests automatizados está totalmente 
demostrada, atopándose en auxe metodoloxías como \emph{TDD (Test Driven 
Development)} ou \emph{BDD (Behaviour Driven Development)} que tratan de 
dirixir o desenvolvemento a través dos tests, que son realizados antes do mesmo.
    
    Durante a primeira parte do desenvolvemento non se aplicóu ningunha de 
estas metodoloxías pero a partir da primeira \emph{release} e da integración 
dos primeiros tests, decidíuse optar por aplicar TDD no desenvolvemento 
realizando probas unitarias nos servicios utilizados.

    \subsection{Fluxo de traballo}
    O fluxo de traballo utilizado dende o primeiro día trata de simular o 
traballo diario de equipo y permite controlar a evolución do código de xeito 
máis ordenado.
    
    Unha nova funcionalidade ou erro son resoltos nunha nova rama 
independente e creada a partir de \emph{master}, tratando que todas estas novas 
funcionalidades sexan independentes entre si.

    Así mesmo tratase de que todos os \emph{commits} sexan o máis independentes 
posibles e todos funcionáis, evitando ter algún que non compile ou que non pase 
os tests.

    Posteriormente realizase unha \emph{Pull Request} a través do mecanismo que 
proporciona o repositorio de código Github, esperando a unha revisión de código 
para ser integrado na rama principal do proxecto.

    Cada certo tempo revísase as \emph{Pull Requests} abertas, analízase o 
codigo e se todo é correcto, aceptase para integrar na rama principal.

    Dende a introducción de tests no proxecto, todo código subido ao 
repositorio é analizado a través dun sistema de integración continua que 
comproba se os cambios engadidos pasan os tests ou non, e avisan por correo 
electrónico do resultado.
