\chapter{Conclusións e traballo futuro}
\minitoc
% \label{chap:Conclusionsetraballofuturo}
% \vspace{0.5cm}

%%%%%%%%%%%%%%%%%%%%%%%%%%%%%%%%%%%%%%%%%%%%%%%%%%%%%%%%%%%%%%%%%%%%%%%%%%%%%%%%
% Objetivo:                        %
%%%%%%%%%%%%%%%%%%%%%%%%%%%%%%%%%%%%%%%%%%%%%%%%%%%%%%%%%%%%%%%%%%%%%%%%%%%%%%%%

  \lettrine{E}{n} este capítulo contaremos as conclusións obtidas da 
realización do proxecto, os recoñecementos obtidos polo mesmo e as liñas de 
traballo futuro que temos plantexadas.

\section{Recoñecementos}

Durante a realización do proxecto tamén se participóu en varios certames e 
concursos, non só coa idea de obter recoñecementos se non tamén co obxectivo de 
difundir o proxecto e buscar colaboradores para a comunidade de software libre 
que estamos a crear.

  \subsection{Finalista no Certamen de Proyectos Libres da UGR}
  O día 6 de Xuño celebróuse na Facultade de Informática da Universidade de 
Granada (UGR) a final do ``Certamen de Proyectos Libres'' que organiza a 
Oficina de Software Libre de dita universidade dende xa fai varios anos.

  VACmatch Mobile convertíuse en un dos 7 finalistas e foi invitado a 
participar presentando o proxecto ante un xurado composto poder diversos 
profesionais do sector, presentación que se realizóu en video ao non 
poder asistir ao evento personalmente.

  \subsection{Premio Universitario de Software Libre}
  Participouse no Concurso Universitario de Software Libre, un certame no que 
competiron ata 75 persoas con máis de 40 proxectos de software libre de todo 
España e no que VACmatch Mobile foi premiado.

  Durante o mes de Maio celebróuse na Facultade de Informática da Universidade 
de Sevilla a fase final nacional na que fomos invitados xunto cos 
outros tres finalistas, a participar e expoñer o noso proxecto ante diversos 
profesionais e empresas do sector.

  Finalmente recibimos o premio ao Mellor Proxecto para Dispositivos Móbiles 
con unha remuneración de 500 \euro{} en metálico e por suposto todos os gastos 
do desplazamento e estancia durante a fase final foron cubertos pola 
organización do concurso.

\section{Conclusións}
O software libre é fundamental na sociedade actual, a inmensa maioría dos 
avances tecnolóxicos baseanse en solucións libres e cada vez son máis os que  
as desenvolven.

  No campo da tecnoloxía aplicada ao deporte, pouco a pouco comézanse a ver as 
primeiras solucións para informatizar a xestión pero, concretamente o 
software libre, atópase aínda con un longo camiño por recorrer e con este 
proxecto conseguimos sentar unhas bases para esto, creando a primeira 
aplicación libre para xestionar actas de encontros deportivos.

  A aplicación foi testeada por usuarios reais polo que está validada 
por profesionáis para ser utiliza en un contexto real, así mesmo atópase 
escrita pensando na extensibilidade, facilitando que os desenvolvedores 
interesados polo proxecto, poidan ampliala de xeito sinxelo e engadir novas 
funcionalidades e, sobre todo, deportes.


\section{Traballo futuro}
Como calquera aplicación actual, este proxecto non é un proxecto totalmente 
concluido xa que certos requisitos inicias non foron aínda implementados e 
múltiples deportes poden ser engadidos.

Tamén gracias a utilizanción de metodoloxías áxiles de desenvolvemento, ao 
longo do proxecto xurdiron novas propostas, das cales algunhas non foron 
ainda implementadas e forman parte de liñas de traballo futuro.

  \subsection{Melloras de desenvolvemento}

    \paragraph{Creación de unha versión de demostración} Sería interesante 
engadir unha funcionalidade para realizar entregas continuas que permita 
automatizar a posta en producción dunha versión do programa como demostración 
para posibles federacións interesadas en coñecer o funcionamento da aplicación.

    \paragraph{Resolución de conflictos} Actualmente cando un elemento se 
modifica ao mesmo tempo en dous lugares a vez (un acta na federación e no 
campo, por exemplo), os conflictos non son resoltos se non que se almacenan 
ambas versións para escoller cal sería a correcta, polo que sería interesante 
proveer aos usuarios de unha interfaz para resolver ditos conflictos de xeito 
sinxelo.

    \paragraph{Integración con VACmatch Web} O punto forte da aplicación sería 
poder integrar o sistema de xestión de competicións de VACmatch coa aplicación 
móbil co fin de simplificar aínda en maior medida o traballo do árbitro e do 
xestor da federación deportiva. 

    \paragraph{Módulo para a xestión multifederación} Outra funcionalidade 
interesante sería a de engadir a posibilidade de realizar a autenticación dos 
árbitros contra diversos hosts de federacións, pensando na opción de que cada 
federación poida ter unha instancia da base de datos propia.

    \paragraph{Módulo de notificacións} Sería interesante que o árbitro 
recibise notificacións sobre cando a aplicación sincroniza novas actas de 
encontros, co fin de facilitarlle coñecer o lugar, data e hora a onde se debe 
desplazar para árbitrar e incluso engadir un sistema de recordatorios dos 
encontros que ten asignados.

  \subsection{Creación de comunidade}
  Nun proxecto de software libre é fundamental dispor dunha comunidade de 
desenvolvedores que axuden a mantelo vivo polo que tamén queremos mencionar 
diversas liñas de traballo a seguir en este punto.

    \paragraph{Hackathons de desenvolvemento.} Organizaranse diversos 
hackathons de desenvolvemento de 1 ou 2 días de duración nos que presentar o 
proxecto e tratar de buscar desenvolvedores para impulsar unha pequena 
funcionalidade ou un pequeno prototipo ao redor de VACmatch co fin de 
introducilos no proxecto.

    \paragraph{Proxectos de Fin de Grao ou Master.} Traballarase con profesores 
e asociacións para promover proxectos de fin de grao e master baseados en 
VACmatch, en lugar de crear pequenas aplicacións a medida para unha asociación 
ou federación, traballar sobre un proxecto grande e múltideporte.

    \paragraph{GPUL Summer of Code.} A asociación GPUL está desenvolvendo un 
programa de apoio a proxectos de software libre para que estudantes 
universitarios colaboren en ditos proxectos durante un verán polo que trataremos 
de propor VACmatch como un dos proxectos no que estos estudantes poidan 
colaborar.


