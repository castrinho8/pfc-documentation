\thispagestyle{empty}
\section*{Agradecementos}

Quero aproveitar este oco para recordar a tódolos que me acompañástedes nesta 
aventura, a tódolos que dende o primeiro día estivéstedes empurrando tamén 
para que este proxecto saíse adiante.

En primeiro lugar quero comezar por Jose Manuel da Federación de Peñas que 
tanto lle dei a vara e tantas horas invertíu de xeito totalmente desinteresado 
en axudarnos.

A Carlota, David e a tódolos emprendedores que coñecín ao longo destes anos 
polos ánimos que me infundiron e ese apoio que sempre atopei en eles.

Por suposto a Fernando Bellas como o meu titor e a Universidade da Coruña 
pola enorme labor docente que me facilitou dispor dos coñecementos para sacar 
adiante este proxecto.

Pero os apoios máis importantes son aqueles que están contigo día tras día 
e que nunca flaquean, aqueles que te empurran incondicionalmente como os 
que meus pais Manolita e Jose, meu irmán Adrian, miña avoa Josefa, meus tíos e 
toda a familia me deron durante tantos anos.

Non vou olvidar os enormes momentos que pasei na facultade con tódolos meus 
compañeiros e en concreto quero agradecerlle a Santi toda a súa axuda neste 
proxecto e as inovidables aventuras que vivimos xuntos nos últimos anos.

A Ali, que é a culpable principal de que conseguise rematar este proxecto, por 
todas esas horas trasnoitando ao meu lado, empurrándome, tranquilizándome e 
conseguindo sacarme unha sonrisa nos peores momentos, sin ela nada sería o 
mesmo.

Aos meus amigos, Revi, Alex, Gato, Javi, Chava e Duda que tantas vivencias 
pasamos xuntos dende moi pequeniños, por esas tardes na biblioteca de 
Intercentros ou esas noites de videoxogos na residencia de estudantes.

E por último quero darlle as gracias a GPUL por todo o que contribuiu a miña 
formación persoal e profesional, a tódolos valores que aprendín aquí e a 
tódalas oportunidades que me está a brindar.

A todos vós, ¡moitas gracias! \\[2cm]

\begin{flushright}
  Pablo Castro Valiño \\
  A Coruña, 20 de xuño de 2016
\end{flushright}
