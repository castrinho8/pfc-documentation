\chapter{Glosario de términos}
\label{chap:glosario-terminos}

%%%%%%%%%%%%%%%%%%%%%%%%%%%%%%%%%%%%%%%%%%%%%%%%%%%%%%%%%%%%%%%%%%%%%%%%%%%%%%%%
% Objetivo: Lista de términos utilizados en el documento,                      %
%           junto con sus respectivos significados.                            %
%%%%%%%%%%%%%%%%%%%%%%%%%%%%%%%%%%%%%%%%%%%%%%%%%%%%%%%%%%%%%%%%%%%%%%%%%%%%%%%%

\begin{description}
  \item [VACmatch] VACmatch é unha plataforma de xestión de competicións deportivas que 
permite realizar todo tipo de trámites coas federacións deportivas de forma electrónica e 
reduce enormemente o traballo que estas deben realizar no seu día a día.
  \item [VACmatch Mobile] é unha aplicación que permite que os árbitros deportivos 
poidan xestionar as actas dos seus encontros de forma electrónica.
  \item [Actas] É o lugar onde se almacena a información sobre un encontro 
deportivo, inclue os equipos, os lugares onde se xogou e o resto de estadísticas de cada 
xogador durante o partido.
  \item [Fichas] Unha ficha é un documento con fotografía incluida que identifica a un 
xogador que compite nunha competición e que debe levar a tódolos encontros para poder 
disputar os partidos.
  \item [Xestor da competición] Persoa encargada da xestión do calendario, da recepción 
das actas dos encontros, da súa revisión, da súa publicación e, en xeral, da xestión dunha 
competición.
  \item [Árbitro] Persoa que se encarga de velar polo complimento do regulamento dun 
deporte durante un encontro e así mesmo debe tomar nota na acta, das estadísticas e dos 
diversos eventos que ocurren nun encontro.
 \item [API] Interfaz de programación de aplicacións que abstrae unha serie de 
funcións para a súa utilización dende outro software.
 \item [Rest] É un tipo de arquitectura de desenvolvemento web que se basea 
no protocolo HTTP coa idea de utilizar os diversos verbos que define o 
protocolo para interactuar cos recursos de unha API.
 \item [Lean Startup] Metodoloxía de desenvolvemento de negocio centrada no 
cliente e na aprendizaxe validada.
 \item [eXtreme Programming] Metodoloxía de desenvolvemento de software que se 
centra na necesidade de adaptarse aos cambios no avance do proxecto co fin de 
dar un punto de vista máis realista que intentar definir todos os requisitos ao 
comezo do proxecto.
 \item [Scrum] Metodoloxía de desenvolvemento de software áxil centrada en 
mellorar a xestión dun proxecto a través dunha serie de prácticas de revisión 
e xestión por iteracións de desenvolvemento chamadas sprints.
 \item [Sprint] Iteración de desenvolvemento dentro das metodoloxías áxiles que 
habitualmente ten unha duración de entre 1 e 3 semanas.
 \item [Release] Versión entregable do produto.
 \item [Pull Request] Petición de integración de unha rama de Git dentro de 
outra. É a forma habitual de colaborar en proxectos que se atopan en GitHub.
 \item [Startup] É un tipo de empresa en construción, habitualmente con 
produto propio, moi áxil e que busca operar con uns costos mínimos e obter 
unhas ganancias de medren exponencialmente.
 \item [Javascript] Linguaxe de programación interpretada, moi habitual para 
desenvolver aplicacións web no lado do cliente, aínda que últimamente está a 
ganar protagonismo no servidor.
 \item [SQLite] Sistema de xestión de bases de datos relacional, moi lixeiro e 
utilizado habitualmente en aplicacións móbiles nativas.
\end{description}
