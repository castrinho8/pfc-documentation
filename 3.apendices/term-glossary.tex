\chapter{Glosario de términos}
\label{chap:glosario-terminos}

%%%%%%%%%%%%%%%%%%%%%%%%%%%%%%%%%%%%%%%%%%%%%%%%%%%%%%%%%%%%%%%%%%%%%%%%%%%%%%%%
% Objetivo: Lista de términos utilizados en el documento,                      %
%           junto con sus respectivos significados.                            %
%%%%%%%%%%%%%%%%%%%%%%%%%%%%%%%%%%%%%%%%%%%%%%%%%%%%%%%%%%%%%%%%%%%%%%%%%%%%%%%%

\begin{description}
  \item [VACmatch] VACmatch é unha plataforma de xestión de competicións deportivas que 
permite realizar todo tipo de trámites coas federacións deportivas de forma electrónica e 
reduce enormemente o traballo que estas deben realizar no seu día a día.
  \item [VACmatch Mobile] é unha aplicación que permite que os árbitros deportivos 
poidan xestionar as actas dos seus encontros de forma electrónica.
  \item [Actas] É o lugar onde se almacena a información sobre un encontro 
deportivo, inclue os equipos, os lugares onde se xogou e o resto de estadísticas de cada 
xogador durante o partido.
  \item [Fichas] Unha ficha é un documento con fotografía incluida que identifica a un 
xogador que compite nunha competición e que debe levar a tódolos encontros para poder 
disputar os partidos.
  \item [Xestor da competición] Persoa encargada da xestión do calendario, da recepción 
das actas dos encontros, da súa revisión, da súa publicación e, en xeral, da xestión dunha 
competición.
  \item [Árbitro] Persoa que se encarga de velar polo complimento do regulamento dun 
deporte durante un encontro e así mesmo debe tomar nota na acta, das estadísticas e dos 
diversos eventos que ocurren nun encontro.
 \item [API Rest]
 \item [Lean Startup]
 \item [eXtreme Programming]
 \item [Scrum]
 \item [Sprint]
 \item [Release]
 \item [Commit]
 \item [Pull Request]
 \item [Startup]
\end{description}
