\newpage
\chapter*{Apéndices}
\thispagestyle{empty}
\newpage

\chapter{Configuración e execución da aplicación.}
\label{chap:compilacion_aplicacion}
VACmatch Mobile é Software Libre e o seu código fonte atópase publicado no 
seguinte repositorio de código en GitHub 
\href{https://github.com/vacmatch/vacmatch-mobile}{
github.com/vacmatch/vacmatch-mobile}.

Para poder executar a aplicación é preciso dispor das seguintes ferramentas:

\begin{itemize}
 \item Git
 \item Gulp
 \item NodeJS
 \item NPM
 \item CouchDB
\end{itemize}

  \section{Execución dende plataforma Web.}

  O primeiro paso é abrir o ficheiro de configuración da aplicación situado en 
\lstinline{src/app/api/config.json} e nel atoparemos a configuración das bases 
de datos remotas para os entornos de desenvolvemento e produción.

  É preciso dispor dunha base de datos CouchDB funcionando ben en local ou ben 
en remoto e no ficheiro de configuración indicado anteriormente podemos 
introducir os datos de conexión e os nomes das coleccións que se van a 
utilizar para a autenticación e para o resto da aplicación.

  Modificando a variable \lstinline{_env} podemos cambiar de entorno e 
a través de \lstinline{_type} podemos seleccionar se a compilación se vai a 
realizar para unha aplicación web ou móbil.

  Unha vez arrancada a aplicación, únicamente é preciso abrir un 
navegador web e acceder a \lstinline{localhost:8080/#/login}

      \lstset{}
      \begin{lstlisting}[caption=Configuración e execución web]
// Instalar dependencias
> npm install
// Compilación
> gulp build
// Arrancar servidor
> gulp run
      \end{lstlisting}

  \section{Execución dende plataforma móbil.}

  Unha vez configurado proxecto, o comando \lstinline{gulp deploy} construe 
e copia a aplicación dentro da carpeta de Apache Cordova.

  Alí débese engadir a plataforma que se desexe utilizar, constuir a 
aplicación para dita plataforma e arrancala de xeito sinxelo.

    \lstset{}
    \begin{lstlisting}[caption=Configuración e execución móbil]
// Instalar dependencias
> npm install

// Compilación
> gulp deploy

// Entrar na carpeta de cordova, engadir plataforma android e construir a 
// aplicación
> cd cordova-app

> cordova platform add android

> cordova build android

// Arrancar en móbil
> cordova run android

// Arrancar nun emulador
> cordova emulate android

    \end{lstlisting}

\thispagestyle{empty}
